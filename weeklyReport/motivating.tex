


\begin{figure}[!htb]
 \begin{minipage}{0.5\textwidth}
\scriptsize 
\begin{tabular}{@{}p{1\textwidth}} 
 \hline 
%  \multicolumn{1}{c}{(A) TextEditor} \\ \hline
  \vspace{-4mm}
\begin{Verbatim}[commandchars=\\\{\}, tabsize=2]
public class TextEditor extends JTextPane \{
\textcolor{purple}{ public UndoAction undoAction = new UndoAction();}
\textcolor{purple}{ public RedoAction redoAction = new RedoAction();}
\textcolor{purple}{ public CompoundUndoManager undo;}
 HashMap<Object, Action> actions = new HashMap<Object, Action>();
 public TextEditor(Workspace workspace) \{
   this.workspace = workspace;
\textcolor{purple}{   undo = new CompoundUndoManager(workspace);}
   actions.put("undo", undoAction);
   actions.put("redo", redoAction);
   \}
 public void discardUndoRedo() \{
    undo.discardAllEdits();
    undoAction.updateUndoState();
    redoAction.updateRedoState();
   \}
\textcolor{purple}{ public class UndoAction extends AbstractAction \{}
\textcolor{purple}{  public void actionPerformed(ActionEvent e) \{}
   try \{
\textcolor{purple}{     undo.undo();}
     updateUndoState();
     redoAction.updateRedoState();
    \} catch (CannotUndoException ex) \{
    \}\}
   public void updateUndoState() \{
    \uwave{setEnabled(undo.canUndo());}
   \}\}
\textcolor{purple}{ public class RedoAction extends AbstractAction \{}
\textcolor{purple}{   public void actionPerformed(ActionEvent e) \{}
    try \{
\textcolor{purple}{      undo.redo();}
      updateRedoState();
      undoAction.updateUndoState();
     \} catch (CannotRedoException ex) \{
     \}\}
      public void updateRedoState() \{
       \uwave{setEnabled(undo.canRedo());}
     \}\}
  \end{Verbatim}
      \vspace{-4mm}
     \\ \hline
\end{tabular} 
\caption{Result No 1: textmash:TextEditor}
\label{fig:textEditor}
\end{minipage}
\end{figure}





\begin{figure}[!htb]
 \begin{minipage}{0.5\textwidth}
\scriptsize 
\begin{tabular}{@{}p{1\textwidth}} 
 \hline 
%  \multicolumn{1}{c}{(B) No. 5.6,8, ConsoleTextEditor} \\ \hline
  \vspace{-4mm}
\begin{Verbatim}[commandchars=\\\{\}, tabsize=2]
public class ConsoleTextEditor extends JScrollPane \{
\textcolor{purple}{  private UndoAction undoAction = new UndoAction();}
\textcolor{purple}{    private RedoAction redoAction = new RedoAction();}
\textcolor{purple}{    private TextUndoManager undoManager;}
    public ConsoleTextEditor () \{
     this.undoManager = new TextUndoManager();
\textcolor{purple}{     doc.addUndoableEditListener(undoManager);}
     undoManager.addPropertyChangeListener(undoAction);
     undoManager.addPropertyChangeListener(redoAction);
\textcolor{purple}{     doc.addDocumentListener(undoAction);}
\textcolor{purple}{     doc.addDocumentListener(redoAction);}
  \}
\textcolor{purple}{ private class RedoAction extends UpdateCaretListener}
  implements PropertyChangeListener \{
\textcolor{purple}{  public void actionPerformed(ActionEvent ae) \{}
\textcolor{purple}{   undoManager.redo();}
   \uwave{setEnabled(undoManager.canRedo());}
   \uwave{undoAction.setEnabled(undoManager.canUndo());}
  \}
    public void propertyChange(PropertyChangeEvent pce) \{
     setEnabled(undoManager.canRedo());
   \} \}
\textcolor{purple}{ private class UndoAction extends UpdateCaretListener }
 implements PropertyChangeListener \{
\textcolor{purple}{   public void actionPerformed(ActionEvent ae) \{}
\textcolor{purple}{    undoManager.undo();}
    \uwave{setEnabled(undoManager.canUndo());}
    \uwave{redoAction.setEnabled(undoManager.canRedo());}
 \}
  public void propertyChange(PropertyChangeEvent pce) \{
    setEnabled(undoManager.canUndo());
  \}\}
  \end{Verbatim}
      \vspace{-4mm}
     \\ \hline
\end{tabular} 
\caption{Result No 5,6,8: groovy: DocumentUndoManagerImpl}
\label{fig:textEditor}
\end{minipage}
\end{figure}





%\section{Motivating Example}
\section{Usage Scenario}
\noindent{\textbf{Add undo and redo action for TextEditor.}}

\begin{table}[ht]
\begin{center}
\caption{Search Result from Code Search Engine}
\label{tab:total}
\scriptsize{
\begin{tabular*}{0.5\textwidth}{@{}c|llrr@{}} \hline
No.&Name&Project&LOC&\# M\\\hline
1&TextEditor&textmash&1270&12\\
2,3&AndroidTextEditor&android&576&9\\
4&DocumentUndoManagerImpl&ide&1227&17\\
5,6,8&ConsoleTextEditor&groovy&321&7\\ 
7&PapyrusCDTEditor&eclipse.papyrus&393&7\\
9,10&AspectEditorContributor&eclipse&88&2\\ \hline
\end{tabular*}
 \label{tab:undoResult}
 \textbf{ \# M} represents the number of methods that contain the query terms. 
 {\bf No.} represents the rank from CSE. Note that CSE may return the same results for multiple times, as the same file may exist in multiple branches. For instance, we regard 5,6,8 as identical with manual inspection. 
 
}
 \end{center}
\end{table}

To further illustrate that reuse task involves in multiple methods and classes, and the reuse task is hard without tool support, I imagine a scenario that user wants to add undo and redo actions for her Java Swing text editor application with undo/redo buttons, inspired by the evaluation task used in~\cite{Murphy:nlConcern11}. Without knowing any APIs, developer  first queries code search engine (CSE) with a free-form query \codefont{`undo redo TextEditor'}. Figure~\ref{fig:undoEditor} illustrates the implementation of `add undo/redo' concern mentioned in~\cite{Murphy:nlConcern11} and blue part represents source code that implements this feature.  Table~\ref{tab:undoResult} shows top 10 results returned from SearchCode CSE. We find that all code examples require multiple methods and classes to implement a feature. We also find that all code examples include other auxiliary features that are not directed related to the undo and redo feature. 

We assume that user provides a seed code example for the reuse task. She selects the first result from CSE. She uses keyword search to locate undo and redo feature in 6 methods and regards them as  seed API calls. Shown in Figure~\ref{fig:undoEditor}, she removes  the other  11 methods (865 LOC) that she regards as irrelevant to her reuse task. She recognizes that she needs to implement an \codefont{UndoAction} and \codefont{RedoAction} which are the subclass of \codefont{AbstractAction}, and overrides their actionPerform() method. In the actionPerformed() methods, she should invokes the \codefont{CompoundUndoManager.undo()} and  \codefont{CompoundUndoManager.redo()} correspondingly. Without knowing anything about \codefont{CompoundUndoManager}, she has to query CSE again for `CompoundUndoManager textmash'. This class consists of 17 methods (221 LOC) and she has to repeat to keyword search again to locate undo and redo feature in this class. With manual inspection, she notices that this \codefont{CompoundUndoManager} is a subclass of \codefont{javax.swing.UndoManager} and it overrides four methods \{\codefont{canUndo, canRedo, undo, redo}\} that are related to the feature. She notices that she actually does not need this \codefont{CompoundUndoManager} and decides to invoke its parent class \codefont{UndoManager}  instead. She also notices that she does not need the \codefont{CannotRedoException} by using \codefont{UndoManager}. 

After she integrates \codefont{UndoManager, UndoAction, and RedoAction} to her context, she tests it and it fails to perform the feature. She has to look at other examples, and she finds the No.5 example returned from CSE.  This example seems promising because it is also an implementation of a TextEditor. She notice that  the API \codefont{doc.addUndoableEditListener (undoManager)} seems the one that she misses now, but again, she has to investigate the class   \codefont{TextUndoManager} again. 

With this example, we illustrate that salient API calls that implement a desired feature are always interleaving with other related elements. To finish a reuse task, developer has to remove irrelevant parts.  However, it  requires significant effort to tease out these auxiliary features without tool support. Since the examples always contain both main feature and auxiliary features,  there seldom exists a perfect example for reuse and developers has to iteratively search for new examples and integrate it to their context until the integration results are examined.     

\section{Problem 3}
Problem: During copy/paste based reuse, developers must remove irrelevant parts and find salient API elements based on the context. 

Hypothesis: Our hypothesis is that having multiple examples of the same kind will help winnow out irrelevant parts, i.e, clustering multiple examples of the same kind and finding the commonality may help remove irrelevant parts. 

%We use the results for  the last three concerns implementation in Table~\ref{tab:concern} found by the code search engine SearchCode to illustrate the Problem 3. 



%  a feature location approach that is similar to~\cite{Denys:FCA12} to identify seed API calls shown in Figure~\ref{fig:undoEditor}. We remove the other  11 methods (865 LOC) that we believe is irrelevant to our reuse task. 

%For each partial program returned from CSE, we select the top k (k=5) methods that are related to the given query. I present No. 1, 5, 6, 8 to show that it requires multiple methods and classes to implement a concern. The other four examples are shown in Appendix.

%In this example, we use feature location approach~\cite{Denys:FCA12} to identify salient API calls, 


















\begin{comment}
\noindent{\textbf{Example collection phase.}} With 7  partial programs at hand, it first extracts class-level structural facts for each code example. Based on a relative match threshold to measure the relative occurrence of the number of matched facts out of all class-level facts,  our tool clusters these examples in three groups using complete linkage technique: 5 examples use classes in Java Swing like `AbstractAction' and `UndoableEditListener', another example uses a `Node' data structure to illustrate how to implement undo/redo actions using command design pattern, and the last example is the javadoc for `UndoManager'.  Our tool starts from the first cluster as it has the highest relative occurrence out of all examples. Based on partial program analysis, we generate a list of type facts to infer types and bind methods.

The common facts our tool extracts include: 
\begin{table}[ht]
\begin{center}
\caption{Common Structural Facts Extracted from the Examples }
\label{tab:total}
\vspace{1mm}
\scriptsize{
\begin{tabular*}{0.5\textwidth}{@{}l|l@{}} \hline
Level&Structural Facts\\\hline
class&subType(Undo*Action, AbstractAction), \\ 
&subType(Redo*Action, AbstractAction), \\
&subType(*Listener, UndoListener)\\ \hline
method&override(actionPerformed(), Undo*Action, AbstractAction), \\
&override(actionPerformed(), Redo*Action, AbstractAction), \\ \hline
statement&init(*,javax.swing.undo.UndoManager), \\
&invoke(UndoManager.undo, actionPerformed(), Undo*Action),\\
& invoke(UndoManager.redo, actionPerformed(), Redo*Action)\\  \hline
\end{tabular*}
 \label{tab:dataset}

We simplify some full names due to space limitation: 

AbstractAction: javax.swing.AbstractAction;

UndoListener: javax.swing.event.UndoableEditListener

actionPerformed():actionPerformed(ActionEvent)
}
 \end{center}
\end{table}



\noindent{\textbf{Example clustering phase.}} After extracting facts for each example, our tool identifies main facts shared in all 5 examples shown in Figure~\ref{fig:fact} part A. It leverages slicing to group related facts. In this example, 3 of them add \codefont{Undo*Action} to \codefont{JMenuItem} by invoking \codefont{JMenuItem.add (ActionEvent)} and support undo and redo actions for Text Editor shown in Figure~\ref{fig:fact} part B. The other two of them invoke \codefont{JButton.addActionListener(ActionEvent)} to add these two actions to \codefont{JButton}, and support \codefont{JApplet} Shape drawing shown in Figure~\ref{fig:fact} part C. After extracting both main facts and related facts, we generate two reuse tasks by combining main facts with each related fact group.  

\noindent{\textbf{Example integration phase.}} Given a context like Figure~\ref{fig:context} part D, our tool identifies the best-fit reuse task based on the number of matched facts in the user's context.  If our tool fails to identify any matched facts in the user's context, it will simply rank the reuse task based on its occurrence in the code examples. Our tool asks for user's selection by presenting the generated main facts and related facts to the user in the form of code snippet. 
After confirmed by users, our tool automatically fills the rest facts that do not exist in the context shown in Figure~\ref{fig:context} part E. 
\end{comment}

