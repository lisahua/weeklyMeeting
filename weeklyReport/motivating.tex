
   


\begin{figure}[!htb]
 \begin{minipage}{0.5\textwidth}
\scriptsize 
\begin{tabular}{@{}p{1\textwidth}} 
 \hline 
%  \multicolumn{1}{c}{(A) TextEditor} \\ \hline
  \vspace{-4mm}
\begin{Verbatim}[commandchars=\\\{\}, tabsize=2]
public class TextEditor extends JTextPane \{
\textcolor{purple}{ public UndoAction undoAction = new UndoAction();}
\textcolor{purple}{ public RedoAction redoAction = new RedoAction();}
\textcolor{purple}{ public CompoundUndoManager undo;}
 HashMap<Object, Action> actions = new HashMap<Object, Action>();
 public TextEditor(Workspace workspace) \{
   this.workspace = workspace;
\textcolor{purple}{   undo = new CompoundUndoManager(workspace);}
   actions.put("undo", undoAction);
   actions.put("redo", redoAction);
   \}
 public void discardUndoRedo() \{
    undo.discardAllEdits();
    undoAction.updateUndoState();
    redoAction.updateRedoState();
   \}
\textcolor{purple}{ public class UndoAction extends AbstractAction \{}
\textcolor{purple}{  public void actionPerformed(ActionEvent e) \{}
   try \{
\textcolor{purple}{     undo.undo();}
     updateUndoState();
     redoAction.updateRedoState();
    \} catch (CannotUndoException ex) \{
    \}\}
   public void updateUndoState() \{
    \uwave{setEnabled(undo.canUndo());}
   \}\}
\textcolor{purple}{ public class RedoAction extends AbstractAction \{}
\textcolor{purple}{   public void actionPerformed(ActionEvent e) \{}
    try \{
\textcolor{purple}{      undo.redo();}
      updateRedoState();
      undoAction.updateUndoState();
     \} catch (CannotRedoException ex) \{
     \}\}
      public void updateRedoState() \{
       \uwave{setEnabled(undo.canRedo());}
     \}\}
  \end{Verbatim}
      \vspace{-4mm}
     \\ \hline
\end{tabular} 
\caption{Result No 1: textmash:TextEditor}
\label{fig:textEditor}
\end{minipage}
\end{figure}





\begin{figure}[!htb]
 \begin{minipage}{0.5\textwidth}
\scriptsize 
\begin{tabular}{@{}p{1\textwidth}} 
 \hline 
%  \multicolumn{1}{c}{(B) No. 5.6,8, ConsoleTextEditor} \\ \hline
  \vspace{-4mm}
\begin{Verbatim}[commandchars=\\\{\}, tabsize=2]
public class ConsoleTextEditor extends JScrollPane \{
\textcolor{purple}{  private UndoAction undoAction = new UndoAction();}
\textcolor{purple}{    private RedoAction redoAction = new RedoAction();}
\textcolor{purple}{    private TextUndoManager undoManager;}
    public ConsoleTextEditor () \{
     this.undoManager = new TextUndoManager();
\textcolor{purple}{     doc.addUndoableEditListener(undoManager);}
     undoManager.addPropertyChangeListener(undoAction);
     undoManager.addPropertyChangeListener(redoAction);
\textcolor{purple}{     doc.addDocumentListener(undoAction);}
\textcolor{purple}{     doc.addDocumentListener(redoAction);}
  \}
\textcolor{purple}{ private class RedoAction extends UpdateCaretListener}
  implements PropertyChangeListener \{
\textcolor{purple}{  public void actionPerformed(ActionEvent ae) \{}
\textcolor{purple}{   undoManager.redo();}
   \uwave{setEnabled(undoManager.canRedo());}
   \uwave{undoAction.setEnabled(undoManager.canUndo());}
  \}
    public void propertyChange(PropertyChangeEvent pce) \{
     setEnabled(undoManager.canRedo());
   \} \}
\textcolor{purple}{ private class UndoAction extends UpdateCaretListener }
 implements PropertyChangeListener \{
\textcolor{purple}{   public void actionPerformed(ActionEvent ae) \{}
\textcolor{purple}{    undoManager.undo();}
    \uwave{setEnabled(undoManager.canUndo());}
    \uwave{redoAction.setEnabled(undoManager.canRedo());}
 \}
  public void propertyChange(PropertyChangeEvent pce) \{
    setEnabled(undoManager.canUndo());
  \}\}
  \end{Verbatim}
      \vspace{-4mm}
     \\ \hline
\end{tabular} 
\caption{Result No 5,6,8: groovy: DocumentUndoManagerImpl}
\label{fig:textEditor}
\end{minipage}
\end{figure}




%\section{Motivating Example}
\section{Usage Scenario}
 To further illustrate that reuse task involves in multiple methods and classes, and the reuse task is hard without tool support, I imagine a scenario that user wants to add undo and redo actions for her Java Swing text editor application with undo/redo buttons, inspired by the evaluation task used in~\cite{Murphy:nlConcern11}.
 
\noindent{\textbf{Query: undo redo TextEditor}}

\begin{table*}[ht]
\begin{center}
\caption{Search Result from Code Search Engine}
\label{tab:total}
\scriptsize{
\begin{tabular*}{1\textwidth}{@{}c|rrrrrl@{}} \hline
No.&Name&Project&LOC&\#M&\#C&Involved objects and API calls\\\hline
1&TextEditor&textmash&1271&12&1&UndoManager, UndoAction, RedoAction\\
4&ConsoleTextEditor&groovy-core&322&7&1&UndoManager, UndoAction, RedoAction\\ \hline
2&AndroidTextEditor&android-sdk&587&9&2& IActionBars.setGlobalActionHandler(), ActionFactory.UNDO, ActionFactory.REDO\\
8&LayoutCanvas&sdk&1721&3&2& IActionBars.setGlobalActionHandler(), ActionFactory.UNDO, ActionFactory.REDO\\
9&IUEditorContributor&buckminster&193&2&2& IActionBars.setGlobalActionHandler(), ActionFactory.UNDO, ActionFactory.REDO\\ \hline
5&PapyrusCDTEditor&papyrus&393&7&4&IActionBars.setGlobalActionHandler(), ITextEditorActionConstants.UNDO\\
6&AspectEditorContributor&eclipse-plugin&89&2&4&IActionBars.setGlobalActionHandler(), ITextEditorActionConstants.UNDO\\
10&WSDLContributor&eclipse&254&2&4&IActionBars.setGlobalActionHandler(), ITextEditorActionConstants.UNDO\\ \hline
3&DocumentUndoImpl&ide&1228&17&3&FileDocumentUndoManager\\ \hline
7&ATETextPane&antlrworks&650&7&5&AbstractUndoableEdit.undo(), AbstractUndoableEdit.redo(),\\\hline
\end{tabular*}
 \label{tab:undoResult}
  {\bf No.} represents the rank from CSE. 
 \textbf{ \#M} represents the number of methods that contain the query terms.  \textbf{ \#C} represents the number of clusters the example belongs to.
}
 \end{center}
\end{table*}


   
 Without knowing any APIs, developer  first queries code search engine (CSE) with a free-form query \codefont{`undo redo TextEditor'}. %Figure~\ref{fig:undoEditor} illustrates the implementation of `add undo/redo' concern mentioned in~\cite{Murphy:nlConcern11} and blue part represents source code that implements this feature.  
Table~\ref{tab:undoResult} shows top 10 results returned from SearchCode CSE. %We find that all code examples require multiple methods and classes to implement a feature. We also find that all code examples include other auxiliary features that are not directed related to the undo and redo feature. 

Before using our tool, she has to manually investigate all code search examples to identify which API calls are frequently used. She selects the first result from CSE and uses keyword search to locate undo and redo feature in 6 methods and regards them as search seed. She started to explore the relationships between these methods.  Shown in Figure~\ref{fig:cluster1} (A), she removes  the other  11 methods (865 LOC) that she regards as irrelevant to her reuse task. She recognizes that she needs to implement an \codefont{UndoAction} and \codefont{RedoAction} which are the subclass of \codefont{AbstractAction}, and overrides their actionPerform() method. In the actionPerformed() methods, she should invokes the \codefont{CompoundUndoManager.undo()} and  \codefont{CompoundUndoManager.redo()} correspondingly. Without knowing anything about \codefont{CompoundUndoManager}, she has to query CSE again for `CompoundUndoManager textmash'. This class consists of 17 methods (221 LOC) and she has to repeat to keyword search again to locate undo and redo feature in this class. With manual inspection, she notices that this \codefont{CompoundUndoManager} is a subclass of \codefont{javax.swing.UndoManager} and it overrides four methods \{\codefont{canUndo, canRedo, undo, redo}\} that are related to the feature. She notices that she actually does not need this \codefont{CompoundUndoManager} and decides to invoke its parent class \codefont{UndoManager}  instead. She also notices that she does not need the \codefont{CannotRedoException} by using \codefont{UndoManager}. 

After she integrates \codefont{UndoManager, UndoAction, and RedoAction} to her context, she tests it and it fails to perform the feature. She has to look at other examples, and she finds the fifth example from CSE is similar to the first example.  Shown in Figure~\ref{fig:cluster1} (B), This example seems similar because it uses an \codefont{TextUndoManager}, and implements \codefont{UndoAction} and \codefont{RedoAction}. She notice that  the API \codefont{doc.addUndoable EditListener(undoManager)} seems the one that she misses now, but again, she has to investigate the class   \codefont{TextUndoManager} again. 

With this example, we illustrate that salient API calls that implement a desired feature are always interleaving with other related elements. To finish a reuse task, developer has to remove irrelevant parts.  However, it  requires significant effort to tease out these auxiliary features without tool support. Since the examples always contain both main feature and auxiliary features,  there seldom exists a perfect example for reuse and developers has to iteratively search for new examples and integrate it to their context until the integration results are examined.     

 Our tool is able to represent code examples returned from CSE in a structured manner. We cluster the examples that share similar API calls and extract the queried features. Shown in Figure~\ref{fig:cluster1} (C), our tool extracts a skeleton of reuse code example for the queried feature. At the same time, we group auxiliary features based on program slicing and lexical similarity so that the user can select the features that she wants, as shown in Figure~\ref{fig:cluster1} (D). 


%We use the results for  the last three concerns implementation in Table~\ref{tab:concern} found by the code search engine SearchCode to illustrate the Problem 3. 



%  a feature location approach that is similar to~\cite{Denys:FCA12} to identify seed API calls shown in Figure~\ref{fig:undoEditor}. We remove the other  11 methods (865 LOC) that we believe is irrelevant to our reuse task. 

%For each partial program returned from CSE, we select the top k (k=5) methods that are related to the given query. I present No. 1, 5, 6, 8 to show that it requires multiple methods and classes to implement a concern. The other four examples are shown in Appendix.

%In this example, we use feature location approach~\cite{Denys:FCA12} to identify salient API calls,





%\noindent{\textbf{Example integration phase.}} Given a context like Figure~\ref{fig:context} part D, our tool identifies the best-fit reuse task based on the number of matched facts in the user's context.  If our tool fails to identify any matched facts in the user's context, it will simply rank the reuse task based on its occurrence in the code examples. Our tool asks for user's selection by presenting the generated main facts and related facts to the user in the form of code snippet. After confirmed by users, our tool automatically fills the rest facts that do not exist in the context shown in Figure~\ref{fig:context} part E. 


