
\begin{figure*}[!htb]
\begin{minipage}{0.47\textwidth}
\scriptsize 
\begin{tabular}{p{1\textwidth}}
 \hline 
 \multicolumn{1}{c}{(A) A type-use pattern with a commonly used name} \\ \hline
%\midrule
/*org.apache.jackrabbit.core.config.RepositoryConfig.java:\href{https://github.com/apache/jackrabbit/blob/a6b82ed15de77f5e883740da725b21cdad059970/jackrabbit-core/src/main/java/org/apache/jackrabbit/core/config/RepositoryConfig.java}{a6b82e}*/
%\vspace*{-4mm}
\begin{Verbatim}[commandchars=\\\{\}, tabsize=2]
1. FileSystem virtualFS;
2. private WorkspaceConfig internalCreateWorkspaceConfig(...)\{
3.   try \{
4.     if (virtualFS != null) \{
5.       virtualFS.close();
6.      \}  \} catch (FileSystemException ignore) \{ \}
\end{Verbatim}
\vspace{-4mm}
\\  \hline
%\midrule
% 
\multicolumn{1}{c}{(B) A method using a field with inconsistent name} \\ \hline
%\vspace{-3mm}
/* org.apache.jackrabbit.core.RepositoryImpl:\href{https://github.com/apache/jackrabbit/blob/a6b82ed15de77f5e883740da725b21cdad059970/jackrabbit-core/src/main/java/org/apache/jackrabbit/core/RepositoryImpl.java}{a6b82e}*/
\begin{Verbatim}[commandchars=\\\{\}, tabsize=2]
1. FileSystem{\bf \hl{repStore}};
2. public void doShutdown() \{
3.   try \{
4.     \uwave{repStore.close()};
5.   \} catch (FileSystemException e) \{
6.        log.error("error while closing file system", e);
7.   \} \} 
\end{Verbatim} 
\vspace{-5mm}
\\  \hline
\multicolumn{1}{c}{(C) A bug fix on the field} \\ \hline
commit \href{https://github.com/apache/jackrabbit/commit/98ae1b3d1481d8713d0f4d8758ef9e19259a7270/?diff=split}{98ae1b}: JCR-1318: Repository Home locked not released despite RepositoryException being thrown.
%
%\vspace{-2mm}
\begin{Verbatim}[commandchars=\\\{\}, tabsize=2]
/* org.apache.jackrabbit.core.RepositoryImpl */
1. FileSystem{\bf \hl{repStore}};
2. public void doShutdown() \{
3.\color{blue}{+  \uwave{if (repStore != null)}} \{
4.  try \{
5.   \uwave{repStore.close()};
6.  \} catch (FileSystemException e) \{
7.    log.error("error while closing file system", e);
8.   \} \}
\end{Verbatim} 
\vspace{-5mm}
\\  \hline
\multicolumn{1}{c}{(D) The deletion of inconsistent name usage} \\ \hline
commit  \href{https://github.com/apache/jackrabbit/commit/f51ed4#diff-9751fd2ad72989dbc11c7caca47e907c}{f51ed4}: JCR-2640: Add repository file system to RepositoryContext. No more need for the separate metaDataStore fs. 
%https://github.com/apache/jackrabbit/commit/9cdb3cf89f11a1a201f758a1bd2f2c93e5718c59#diff-9751fd2ad72989dbc11c7caca47e907c
 % https://github.com/apache/jackrabbit/commit/9cdb3cf89f11a1a201f758a1bd2f2c93e5718c59#diff-9751fd2ad72989dbc11c7caca47e907c
%\vspace{-2mm}
\begin{Verbatim}[commandchars=\\\{\}, tabsize=2]
/* org.apache.jackrabbit.core.RepositoryImpl */
1.\color{red}{- FileSystem{\bf \hl{repStore}}};
2. public void doShutdown() \{
3.\color{red}{-  \uwave{if (repStore != null)} \{}
4.  try \{
5.\color{blue}{+  context.getFileSystem().close();}
6.  \} catch (FileSystemException e) \{
7.    log.error("error while closing file system", e);
8.   \} \}
9. public Thread newThread(Runnable r) \{
10.\color{blue}{+ FileSystem repStore = repConfig.getFileSystem();}
11.\color{blue}{+ context.setFileSystem(repStore);}
12. \}
\end{Verbatim} 
commit \href{https://github.com/apache/jackrabbit/commit/9cdb3c#diff-9751fd2ad72989dbc11c7caca47e907c}{9cdb3c}: (completely deletes this inconsistent name) 

JCR-2640: Streamline namespace and node type registry creation
\begin{Verbatim}[commandchars=\\\{\}, tabsize=2]
13. public Thread newThread(Runnable r) \{
14.\color{red}{- FileSystem repStore = repConfig.getFileSystem();}
15.\color{red}{- context.setFileSystem(repStore);}
16.\color{blue}{+ context.setFileSystem(repConfig.getFileSystem());}
17. \}
\end{Verbatim} 
\vspace{-5mm}
\\  \hline
A \codefont{FileSystem} object receives a method call \codefont{FileSystem.close()} in a \codefont{try-catch} block. The majority of such objects with similar type-use pattern contains a term \codefont{FS} in their name except for \codefont{repStore} in Part (B). A developer fixed a bug in this method with an inconsistent name at line 3 in Part (C), and removed this inconsistent name from the type-use pattern in commit \href{https://github.com/apache/jackrabbit/commit/f51ed4#diff-9751fd2ad72989dbc11c7caca47e907c}{f51ed4}. This inconsistent name was completely deleted in commit \href{https://github.com/apache/jackrabbit/commit/9cdb3c#diff-9751fd2ad72989dbc11c7caca47e907c}{9cdb3c}. 
\end{tabular}
\end{minipage}
\hfill
\begin{minipage}{0.47\textwidth}
\scriptsize 
\begin{tabular}{p{1\textwidth}}
 \hline 
 \multicolumn{1}{c}{(A) A type-use pattern with a commonly used name} \\ \hline
/* org.apache.james.transport.mailets.Retry.java:\href{https://github.com/apache/james/blob/ed9bb43ec9f5b25f799e38c3a4b6e0d5053fc72f/mailets-function/src/main/java/org/apache/james/transport/mailets/Retry.java}{ed9bb4} */
%\vspace*{-4mm}
%snapshot link: https://github.com/apache/james/blob/ed9bb43ec9f5b25f799e38c3a4b6e0d5053fc72f/mailets-function/src/main/java/org/apache/james/transport/mailets/Retry.java
\begin{Verbatim}[commandchars=\\\{\}, tabsize=2]
1. private SpoolRepository retryRepository;
2. private MultipleDelayFilter delayFilter;
3.  public void run() \{
4.    Mail mail = retryRepository.accept(delayFilter);
5.    String key = mail.getName();
6.    if (retry(mail)) \{
7.   	retryRepository.remove(key);
8.     \} else \{
9.      retryRepository.store(mail);
10.    retryRepository.unlock(key);
11.    \}\}
\end{Verbatim}
\vspace{-4mm}
\\  \hline
%\midrule
\multicolumn{1}{c}{(B) A method using a field with inconsistent name } \\ \hline
%\vspace{-3mm}
/*org.apache.james.transport.mailets.RemoteDelivery.java:\href{https://github.com/apache/james/blob/ed9bb43ec9f5b25f799e38c3a4b6e0d5053fc72f/mailets-function/src/main/java/org/apache/james/transport/mailets/RemoteDelivery.java}{6aa468}*/
%snapshot link: https://github.com/apache/james/blob/ed9bb43ec9f5b25f799e38c3a4b6e0d5053fc72f/mailets-function/src/main/java/org/apache/james/transport/mailets/RemoteDelivery.java
\begin{Verbatim}[commandchars=\\\{\}, tabsize=2]
1. private SpoolRepository \hl{outgoing}; // The spool of outgoing mail
2. private MultipleDelayFilter delayFilter;
3. public void run() \{
4.  Mail mail = (Mail) \uwave{outgoing.accept(delayFilter)};
5.  String key = mail.getName();
6.  if (deliver(mail, session)) \{
7.   \uwave{outgoing.remove(key);}
8.    \} else \{
9.    \uwave{outgoing.store(mail);}
10.   \uwave{outgoing.unlock(key);}
11.  \}\}
\end{Verbatim} 
\vspace{-5mm}
\\  \hline
% refactoring: https://github.com/apache/james/commit/54cccd3a67dcd
\multicolumn{1}{c}{(C) A refactoring on the field} \\ \hline
commit \href{https://github.com/apache/james/commit/54cccd3a67dcd}{54cccd}: Minor changes to RemoteDelivery and Retry to make them more similar (I'd like to extract common code) (related to JAMES-874).
%\vspace{-2mm}
\begin{Verbatim}[commandchars=\\\{\}, tabsize=2]
/* org.apache.james.transport.mailets.RemoteDelivery.java */
// Repository used to store messages that will be delivered.
1. private SpoolRepository\textcolor{blue}{\underline{\underline{\bf workRepository}}}; 
2. private MultipleDelayFilter delayFilter;
3. public void run() \{
4.  Mail mail = (Mail)\textcolor{blue}{\bf \uwave{workRepository.accept(delayFilter)}};
5.  String key = mail.getName();
6.  if (deliver(mail, session)) \{
7.   \textcolor{blue}{\bf \uwave{workRepository.remove(key);}}
8.    \} else \{
9.   \textcolor{blue}{\bf \uwave{workRepository.store(mail);}}
10.  \textcolor{blue}{\bf \uwave{workRepository.unlock(key);}}
11.  \}\}
\end{Verbatim} 
\vspace{-5mm}
\\  \hline
Part (A) represents a type-use pattern for \codefont{SpoolRepository} object from line 4 to line 10. The objects in similar type-use patterns are commonly named as \codefont{*Repository} except for \codefont{outgoing} in Part (B).  This inconsistent object name was renamed as \codefont{workRepository} in commit \href{https://github.com/apache/james/commit/54cccd3a67dcd}{54cccd} to enforce naming inconsistency. \todo{Lisa: I prefer the first example as there is no bug fix related to this example, although the inconsistent name was renamed.}
\end{tabular}
%\caption{Inconsistent field identified by type-use pattern analysis}
%\label{fig:usageExample}
\end{minipage}
%\end{figure}
%\begin{figure}[!htb]

\caption{Inconsistent field identified by type-use pattern analysis}
\label{fig:usageExample}
\end{figure*}


