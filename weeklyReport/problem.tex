\section{Program Repair}




\section{Improvement Idea} 

\noindent{\bf{Use NL to guide the search for program repair. }}  I manually check 105 defects/changes and their generated patches, I find that a) some plausible but not correct patches for defects are unrelated to bug descriptions in issue tracking system, b) there are a number of defects (30/69) that current auto-repair tools cannot generate any patches (neither plausible nor correct), but have the potential to be (partially) repaired with more 'specification' given by bug description. I argue that apart from passed and failed test cases, user is able to provide bug description in natural language that is also a special kind of specification to guide the search for the program repair.

For example, php-309453-309456 (commit 8deb11c, bug #54291), none of auto-repair tools generate any fix for this bug in the benchmark, these tools include SPR [1](see its generated patches), Prophet [2], GenProg [4], AE [5], Kali [3], RSRepair [6] based on the evaluation, (PAR is unknown) .

This bug is caused by a null pointer exception and developer fixes it with correct exception handling. The reason why none of these tools handle this defect is because, 1) SPR fails to use control-flow transformation schema to this two-statement fix due to large search space, 2) Prophet does not learn this fix based on other fix patches,  3) There is no similar fix statement around thus GenProg, AE, and RSRepair fail to find the repair, 4) it adds code thus Kali that relies on statement deletion cannot fix it. In summary, all of these tools rely sole on test suite or cross-application history, but fail to understand what kind of defect user finds and what kind of repair user prefer.

Bug #54291 Crash in spl_filesystem_object_get_path
Description:
------------
The attached code crashes on PHP5.3 SVN. Since this is pretty sure only a null-pointer deref, reporting this as public (non-security related)...
Actual result:
--------------
==2043== Invalid read of size 8
==2043==    at 0x6586DB: spl_filesystem_object_get_path (spl_directory.c:168)
...
The underlined line causes an null-pointer exception: intern->u.dir.dirp is null. Developer added exception handling in their fix. 
PHPAPI char* spl_filesystem_object_get_path(spl_filesystem_object *intern, int *len TSRMLS_DC) /* {{{ */
{
#ifdef HAVE_GLOB
     if (intern->type == SPL_FS_DIR) {
           if (php_stream_is(intern->u.dir.dirp ,&php_glob_stream_ops)) {
                  return php_glob_stream_get_path(intern->u.dir.dirp, 0, len);
                    }
                     }
#endif
                      if (len) {
                            *len = intern->_path_len;
                             }
                              return intern->_path;
} /* }}} */
static void spl_filesystem_dir_open(spl_filesystem_object* intern, char *path TSRMLS_DC){
    if (EG(exception) || intern->u.dir.dirp == NULL) {  
            intern->u.dir.entry.d_name[0] = '\0';      intern->u.dir.entry.d_name[0] = '\0';
             +  if (!EG(exception)) {
                  +   /* open failed w/out notice (turned to exception due to EH_THROW) */
                   +   zend_throw_exception_ex(spl_ce_UnexpectedValueException, 0
                    +    TSRMLS_CC, "Failed to open directory \"%s\"", path);
                    +  } 
                    I hypothesize that bug reports such as "Crash in spl_filesystem_object_get_path " should give us hint based on NL semantic parsing: <type>crash</type> in <place>spl_filesystem_object_get_path</place>

                    And this information should guide us to search for a better program repair.

                    Q: Any prior works that use NL in program repair?
                    A: From the best of my knowledge (and based on a latest Program Repair Bibliography [13]), we don't have any tools that try to understand the natural language meaning in bug report and use it in program repair.
                    Q: Why NL -semantic parsing can provide more information that test cases?
                    A: NL in API document has been used to generate specification [17], indicating that NL has the potential to generate specification. Bug report is different from API document, but the idea is similar.
                    Q: Why choosing semantic praising ?
                    A: NL-based Software Analysis has been applied to many applications [18], yet most of them are confined to understanding the POS tagging properties (such as verb, noun) [17], and forget to understand the semantic relationship between each entity, e.g., <type>crash</type> in <place>spl_filesystem_object_get_path</place>
                    means that there is an error type--'crash' happened in the method spl_filesystem_object_get_path
