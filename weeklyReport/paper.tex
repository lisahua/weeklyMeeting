%\documentclass[natbib,preprint]{sigplanconf}
% \documentclass[conference]{IEEEtran}
\documentclass{sig-alternate}
\usepackage{fancybox}
\usepackage{cite}
\usepackage{color,soul}
\usepackage{hyperref}
\usepackage{algorithmic}
\usepackage{flushend}
\usepackage{fancyvrb}
\usepackage[normalem]{ulem}
\usepackage{graphicx}
\usepackage[lined, algonl, ruled, boxed]{algorithm2e} 
\usepackage{amsmath}
\usepackage{amssymb} 
\usepackage{graphicx}
\usepackage{verbatim}
\usepackage{booktabs}
\usepackage{multirow}
\usepackage{listings}
\usepackage{subfigure}
\usepackage{latexsym}
\usepackage{xspace}
\usepackage[usenames,dvipsnames]{xcolor}

% \usepackage{fancybox}
% \usepackage[usenames,dvipsnames,svgnames,table]{xcolor}
% \captionsetup[table]{skip=10pt}

\newcommand{\tool}{\textsc{Clipboard}\xspace}
\newcommand{\fixme} [1] {\textcolor{red}{{\it FIXME}: #1}}
\newcommand{\delete} [1] {\textcolor{red}{#1}}
\newcommand{\add} [1] {\textcolor{blue}{#1}}
% \newcommand{\javac}{javac\xspace} % This package lets you punctuate \javac normally and get good spacing, e.g., \javac.  gives you: javac.
\newcommand{\codefont}[1]{\footnotesize{\texttt{#1}}\normalsize}
\newcommand{\todo} [1]{\textcolor{blue}{{\sf TODO}: #1}}
\newcommand{\hlc}[2][yellow]{ {\sethlcolor{#1} \hl{#2}} }


% \lstset{frame=tb,
%   language=Java,
%   showstringspaces=false,
%   columns=flexible,
%   basicstyle={\ttfamily\footnotesize},
%   numberstyle=\tiny\color{gray},
%   keywordstyle=\color{blue},
%   commentstyle=\color{dkgreen},
%   stringstyle=\color{mauve},
%   breaklines=true,
%   breakatwhitespace=true,
%   tabsize=3
% }



\begin{document}

%\conferenceinfo{Mystery'09,} {January 1, 2009, Austin, TX.}
%\CopyrightYear{2009}
%\copyrightdata{2009}


%\title{How Universal are Naming Rules?}

\title{Weekly Meeting } %

% \numberofauthors{3} 
% \author{
% You can go ahead and credit any number of authors here,
% e.g. one 'row of three' or two rows (consisting of one row of three
% and a second row of one, two or three).
%
% The command \alignauthor (no curly braces needed) should
% precede each author name, affiliation/snail-mail address and
% e-mail address. Additionally, tag each line of
% affiliation/address with \affaddr, and tag the
% e-mail address with \email.
%

\numberofauthors{1} %  in this sample file, there are a *total*
% of EIGHT authors. SIX appear on the 'first-page' (for formatting
% reasons) and the remaining two appear in the \additionalauthors section.
%
%\author{
% You can go ahead and credit any number of authors here,
% e.g. one 'row of three' or two rows (consisting of one row of three
% and a second row of one, two or three).
%
% The command \alignauthor (no curly braces needed) should
% precede each author name, affiliation/snail-mail address and
% e-mail address. Additionally, tag each line of
% affiliation/address with \affaddr, and tag the
% e-mail address with \email.
%
% 1st. author
%\alignauthor
%Lisa Hua$^{\dag}$ Na Meng$^{\dag}$ Miryung Kim$^{\ast}$ Kathryn S. McKinley$^{\ddag}$\\
%\affaddr{$^{\dag}$\normalsize{The University of Texas at Austin~~~~~~~$^{\ast}$University of California, Los Angeles~~~~~~~$^{\ddag}$Microsoft Research}}\\
%\email{\normalsize{\{lisahua@, mengna09@cs\}.utexas.edu, miryung@cs.ucla.edu, mckinley@microsoft.com}}
%}

\maketitle

\section{Related Work}

\paragraph{Corpus-wide Naming Rules}  Abebe et al.\/ investigate identifier quality  based on a catalog of Lexicon Bad Smells~\cite{abebe:wcre12}. 
Caprile and Tonella leverage pre-defined syntax and identifier dictionary to replace terms with their synonyms to ensure naming consistency~\cite{caprile:reconstruct}. Dei{\ss}enb{\"{o}}ck and Pizka proposed a formal model for concise and consistent naming, which requires a mapping from a concept domain to a lexical domain~\cite{ccnames:Pizka06}. Lawrie et al.\/ use hard-coded naming rules such as \textit{an identifier should not include another identifier in its name}~\cite{syntacticcc:SCAM06}.  Arnaoudova et al.\/ define a family of linguistic anti-patterns such as \textit{set* methods should not return an object} and investigate misunderstanding caused by anti-patterns~\cite{antipattern:CSMR13}. None of these approaches detect project-specific naming inconsistency.  
Allamanis et al. \/ build a tool called Naturalize to learn and enforce coding conventions~\cite{birdfse14:convension}. It uses n-gram based statistical natural language processing model to suggest natural identifier names and formatting conventions at the level of type names, method calls, and variable names. 
%  The only overlapping target in scope is the variable name recommendations reported by Naturalize. We run Naturalize on the same 39 projects using its default settings and compare variable name reports. We use a few examples to illustrate the difference between \niche and Naturalize. Naturalize misses inconsistent object names such as `\codefont{repStore}', which violates the commonly used name pattern (`\codefont{*FS}') found by type-use pattern analysis in \niche. On the other hand, Naturalizes suggests renaming `\codefont{local\_folder}' to `\codefont{localFolder}'. \niche does not produce such naming suggestion, as it focuses on checking name consistency based on the code's semantic functionality.
H{\o}st et al. \/ build a tool called Lancelot to find poor method names by comparing syntactic code features, such as `\textit{contains a loop}'~\cite{nameBug:ECOOP09} .  
%The only overlapping scope for Lancelot and \niche is the inconsistent method names found by \niche and the `\textit{method naming bugs}' from Lancelot. Lancelot misses inconsistent method names such as `\codefont{setSyncFlag}' which does not change the field value because it checks syntactic features only without considering the purity of a method. Lancelot reports  `\codefont{setExit}' method as a naming bug, since it returns the object itself (\codefont{return this}), but in this case, \niche classifies it as consistent because the method is impure. 
Buse and Weimer develop a readability metric for Java by training data from human annotators~\cite{buse:readability}. They find that their readability metric exhibits a significant level of correlation with static analysis warnings found by FindBug~\cite{FindBug:OOPSLA06}. Though their work connects the readability of identifier names to  static analysis warnings, they do not study the correlation between inconsistent names and bug fixes. 

\paragraph{Defect Study}  Abebe et al.\/ find correlation between naming rule violations and  coupling metrics~\cite{abebe:wcre12,ckmetrics:oopsla91}. Butler et al.\/ find correlation between FindBug's static analysis warnings and violations of hard-coded rules~\cite{FindBug:OOPSLA06,butler:csmr10}. Since static analysis tools suffer from 30\% to 100\% false positive rates~\cite{kremenekWarn:fse04}, we cannot conclude whether naming inconsistency is correlated with actual defects. Boogerd et al.\/ find that only 10 out of 88 hard-coded naming rules have positive correlation with defect density~\cite{boogerd:icsm08}. We differ from prior studies by mining rules from a corpus and studying how these rule violations relate to actual defects. Our study also finds that the confusion and misunderstanding caused by inconsistent name may propagate to their callers and that the lifetime of inconsistent names is shorter than the rest. 

%\paragraph{Mining Usage Patterns} 
 
\paragraph{Feature Location} Poshyvanyk et al. \/ use information retrieval approach to locate a queried feature in source code~\cite{Denys:FCA12}. They  evaluate the similarity between documents and user query and cluster the source code based on formal concept analysis.  Portforlio~\cite{Portfolio:DenysICSE11} and Export~\cite{Export:DenysASE13} identify related functions by combining both latent structure similarity and lexical information similarity. Our feature location approach is similar to~\cite{Portfolio:DenysICSE11} yet we focus on suggesting structured implementation for the feature rather than identify feature location.
Rastkar et al. \/ summarize the structure of multiple instances of a crosscutting concern in natural language, yet they only extract structural facts in the level of method signature and class hierarchy~\cite{Murphy:nlConcern11}. We make it one step further to suggest structured feature implementation based on  user query.

\paragraph{Code Search} Our example clustering approach is similar to some prior works that extract representative examples for specific APIs or user query. 
MAPO~\cite{MAPO:ECOOP09} leverages frequent call sequences to cluster the usage of specific APIs and rank abstract usage patterns based on the context similarity. Buse et al\/\cite{Buse:apiICSE12} propose to generate abstract API usages by synthesizing code examples using symbolic execution for a particular API. Different from these works that generate abstract usage patterns for specific API or data type, SNIFF~\cite{sniff:Sen09} performs type-based intersection of code chunks based on the keywords in the free-form query and cluster the common part of the code chunks for concrete code examples. However, these works   only focus on providing code examples based on the popularity or textural relevance while developers have to manually resolve structural dependencies before reusing the examples.  MUSE~\cite{MUSE:MarcusICSE15} addresses this limitation using slicing to generate concrete usage examples and selects the most representative ones based on the popularity and readability while  Keivanloo et al\/\cite{spotWork:ICSE14} uses clone detection to cluster examples involving loops and conditions.  There exists a number of code example suggestion tools that recommend call chains~\cite{Mandelin:jungloid05, parseWeb:ASE07, Xsnippet:OOPSLA06}  or contexts~\cite{Holmes:structural05, Prompter:MSR14}. But none of  these tools presents code example in a structured manner and identify  common features for each code example cluster, which helps developers to reuse the code examples. 

%only recommend code examples in the method level which make them insufficient for  reuse tasks across multiple classes.   We make it one step further to  identify  both common features and alternative features.

\paragraph{Code Reuse}  Although some approaches advocate refactoring code rather than reuse code~\cite{fowler:refactoring}, recent researches have found that these kind of `clone' cannot be easily refactored~\cite{Kim:cloneGenealogy05} and have to be modified to meeting requirements in new context~\cite{Selby:largeReuse05}.  Jigsaw~\cite{Cottrell:jigsaw08} supports small-scale integration of source code into target system  between the example and target context. Based on its ancestor~\cite{Cottrell:generalize07} that identifies structural correspondence based on AST similarity, it greedily matches each element between two contexts, transforms correspondent elements to the target context, and simply copies the source element to the target if it does not correspond with any element in the target. Unfortunately, developer has to provide source and target to enable a one-to-one transformation and resolve all dependencies when pasting code to the target.  Our approach overcomes these two limitations: we extract common functionalities from multiple examples and  identifies how related elements interact with main features.  Other works on code reuse include
Gilligan~\cite{Holmes:reuse07} and Procrustes~\cite{Holmes:ASE09} which try to address the problem of source code integration in the context of medium or large-scale reuse tasks. They automatically suggest program elements that are easy to reuse based on structural relevance and cost of reuse in the source context, and guide users to investigate and plan a non-trivial reuse task. They assume that developers have a perfect example at hand, and they can finish the reuse task by resolving all dependency conflicts and integrating the example to the desired context.  We don't have this assumption and our tool helps users identify the best-fit example. Our idea of leveraging multiple examples to discover commonality is similar to LASE~\cite{LASE:ICSE13}, which applies similar but not identical changes to multiple code locations based on context similarity.   Our approach works in a similar manner of Programing-by-Example in the context of code reuse task. LASE requires users to specify multiple input examples and search for the third one, while our approach uses free-form queries to obtain hundreds of examples and cluster them based on their common features.

%focuses on a task-based code reuse across different methods or even different classes, while LASE is confined to the systematic edit within a single method and requires users to specify all input examples. 


%However, we note that it is not easy to identify a good example as an example is always interleaving with other auxiliary features that should not be integrated. We observe that it is equally difficult, if not more so, to distinguish the major functionality and auxiliary ones from multiple examples than to identify related elements in a pragmatic reuse plan. We target the problem to identify the major features across different reusable examples and leverage Procrustes to evaluate the cost of reuse when recommending the best-fit reusable plan.



\section{Problem}  
To obtain necessary functionality, developers often use third-party libraries. Due to the complexity of current software systems, the dependency issue arises when the system depend on different and incompatible versions of the same library. This issue, called  `dependency hell'~\cite{wiki:hell} may break other dependencies or push the problem to another set of libraries. When developers try to introduce a new library or upgrade existing ones, then other applications on their system might suddenly break as the newly-introduced libraries are not backward-compatible to the existing libraries. We use an example to illustrate how the `dependency hell' causes a build error, how it is localized, how it is fixed in the next session. 

%\noindent{\bf{Problem }}

\section{Example}


 \begin{figure}[!htb]
 \includegraphics{akka.jpeg}
 \end{figure}
\noindent{\bf{What is the `dependency hell'?}}

\begin{comment}
\begin{figure}[!htb]
 \begin{minipage}{0.5\textwidth}
\scriptsize 
\begin{tabular}{@{}p{1\textwidth}} 
 \hline 
  \multicolumn{1}{c}{(A) Common facts} \\ \hline
  \vspace{-4mm}
\begin{Verbatim}[commandchars=\\\{\}, tabsize=2]
import javax.swing.undo.UndoManager;
import javax.swing.event.UndoableEditListener;
import javax.swing.AbstractAction;

public class Foo \{
  public void run() \{
   UndoManager undoManager = new UndoManager();
  \}\}
  class UndoListener implements UndoableEditListener \{
  public void undoableEditHappened(UndoableEditEvent e) \{
    undoManager.addEdit(e.getEdit());
  \}\}
class UndoAction extends AbstractAction \{
   public void actionPerformed(ActionEvent e) \{   
     undoManager.undo();  
  \}\}
class RedoAction extends AbstractAction \{
  public void actionPerformed(ActionEvent e) \{
     undoManager.redo();
  \}\}
\end{Verbatim}
\vspace{-4mm}
 \\ \hline
  \multicolumn{1}{c}{(B) Auxiliary facts: Menu} \\ \hline
    \vspace{-4mm}
\begin{Verbatim}[commandchars=\\\{\}, tabsize=2]
import com.sun.java.swing.*;

public class Foo \{
  public void run() \{
   UndoManager undoManager = new UndoManager();
   new TextComposite().getDocument().addUndoableEditListener(new UndoListener());
   new JMenuItem(new UndoAction());
   new JMenuItem(new RedoAction());
  \}\}
 \end{Verbatim}
   \vspace{-4mm}
  \\ \hline
   \multicolumn{1}{c}{(C) Auxiliary facts: Button} \\ \hline
  \vspace{-4mm}
\begin{Verbatim}[commandchars=\\\{\}, tabsize=2]
import javax.swing.undo.UndoableEditSupport;
import com.sun.java.swing.Button;

public class Foo \{
  public void run() \{
   UndoManager undoManager = new UndoManager();
   new Button.addActionListener(new UndoAction());
   new Button.addActionListener(new RedoAction());
   new UndoableEditSupport().addUndoableEditListener(new UndoListener());
  \}\}
  \end{Verbatim}
        \vspace{-4mm}
     \\ \hline
\end{tabular} 
\caption{Common features and auxiliary features extracted from informal resources}
\label{fig:fact}
\end{minipage}
\end{figure}
\end{comment}


\begin{figure}[!htb]
 \begin{minipage}{0.5\textwidth}
\scriptsize 
\begin{tabular}{@{}p{1\textwidth}} 
 \hline 
  \multicolumn{1}{c}{(A) User's context} \\ \hline
  \vspace{-4mm}
\begin{Verbatim}[commandchars=\\\{\}, tabsize=2]
import com.sun.java.swing.*;

public class MyTextEditor \{
  public void init() \{
    JFrame frame = new JFrame("Undo Sample");
    frame.setDefaultCloseOperation(JFrame.EXIT_ON_CLOSE);
    JTextArea textArea = new JTextArea();
   JButton undoBtn_;
   //add undo and redo action to text editor
  \}\} 
\end{Verbatim}
    \vspace{-4mm}
     \\ \hline
  \multicolumn{1}{c}{(E) Expected reuse plan based on the query `undo redo TextEditor'} \\ \hline
\begin{Verbatim}[commandchars=\\\{\}, tabsize=2]
import com.sun.java.swing.*;
import javax.swing.undo.UndoManager;
import javax.swing.event.UndoableEditListener;
import javax.swing.AbstractAction;

public class MyTextEditor \{
    JFrame frame = new JFrame("Undo Sample");
    frame.setDefaultCloseOperation(JFrame.EXIT_ON_CLOSE);
    JTextArea textArea = new JTextArea();
    JButton undoBtn_;
\textcolor{blue}{    JButton redoBtn_; }
  
\textcolor{blue}{   public void init() \{}
   //add undo and redo action to text editor
\textcolor{blue}{   UndoManager undoManager = new UndoManager(); }
\textcolor{blue}{   textArea.getDocument().addUndoableEditListener(new UndoListener());}
\textcolor{blue}{   undoBtn_ = new Button();}
\textcolor{blue}{   undoBtn_.addActionListener(new UndoAction());}
\textcolor{blue}{   redoBtn_ = new Button().addActionListener(new RedoAction());}
\textcolor{blue}{   redoBtn_.addActionListener(new RedoAction());}
\textcolor{blue}{   \}}
\textcolor{blue}{ private class UndoListener implements UndoableEditListener \{}
\textcolor{blue}{  public void undoableEditHappened(UndoableEditEvent e) \{}
\textcolor{blue}{    undoManager.addEdit(e.getEdit());}
\textcolor{blue}{  \}\}}
\textcolor{blue}{private class UndoAction extends AbstractAction \{}
\textcolor{blue}{  public void actionPerformed(ActionEvent e) \{   }
\textcolor{blue}{    undoManager.undo(); }
\textcolor{blue}{  \}\}}
\textcolor{blue}{private class RedoAction extends AbstractAction \{}
\textcolor{blue}{  public void actionPerformed(ActionEvent e) \{}
\textcolor{blue}{     undoManager.redo();}
\textcolor{blue}{  \}\}\}}
  \end{Verbatim}
      \vspace{-4mm}
     \\ \hline
\end{tabular} 
\caption{Scenario 1: add undo/redo to a TextEditor}
\label{fig:undoEditor}
\end{minipage}
\end{figure}







We illustrate the building error caused by `dependency hell' in a Maven project shown in Figure~\ref{fig:shifu}. As shown in part A, the parent project uses akka 2.1.1 to initialize \codefont{ActorRef} (underline) by creating a new \codefont{Props} object. The developer Alice is asked to implement a new sub-module named as SparkPlugin using Spark framework. Without knowing that Spark is dependent on akka-2.2.3, she implements the entire sub-module with Spark-1.0.0, writes the tests, and fully tests the single submodule before integrating with the parent project. However, she encounters the building error shown as part (C) (The complete building error is shown at~\cite{shifu}). 

\noindent{\bf{How to localize `dependency hell'?}} 

Starting from the \codefont{NoSuchMethodException}, she investigates the \codefont{akka.remote.RemoteActorRefProvider.<init>}  in akka-2.2.3 as declared in Spark's pom.xml and finds the static \codefont{Props.create} method shown in part (D). Alice get confused on the \codefont{NoSuchMethodException} and she has to use step-by-step debugging. She finally notices that the maven build system mistakenly uses akka-2.1.1 inherited from parent project, rather than the akka-2.2.3 that is required for Spark library. 

\noindent{\bf{How to fix `dependency hell'?}}

  Alice has several options to fix this bug as below:
\begin{enumerate}
\item change akka version in parent class to 2.2+, which is not feasible as the parent project heavily uses akka and it requires tremendous effort for upgrading.
\item  exclude all transitive dependencies inherited from parent project with the feature provided by Maven 3.2.2+~\cite{maven:note}, aiming to resolve dependency hell~\cite{maven:hell}. This approach is not feasible as well because SparkPlugin itself relies on the parent projects and Alice has to re-import all dependencies that the submodule uses.
\item include both akka versions and enforce Spark to use akka-2.2.3. This approach is similar to the well-known side-by-side  mechanism used in NIX package manager for Linux-based system~\cite{nix}. This might fix the build error, but Alice is concerned that compiling both versions might increase the building time of the entire system. 
\item exclude akka-2.1.1 from the dependency declaration of Spark-1.0.0. Alice decides to use this solution as it won't have ripple effect on the rest of the system.
\end{enumerate}


%To ensure that there does not exist any other transitive dependency that causes dependency hell, she tries maven-enforcer plugin which have the rule to ban all transitive dependencies~\cite{maven:enforcer}, yet the plugin outputs more than 100 conflicts. Alice gives up to kill all conflicts as (1) It is costly to exclude all conflicts, (2) it makes the maven hard to maintain, and (3) most of them might not be harmful for the system. 

Using this example, we illustrate that  it is not easy to  localize dependency hell without knowing the entire dependency graph and it can be error-prone when trying to fix the error.

\section{Related Work}

When two versions of the same library occur, Maven selects the closest library  in the tree of dependencies by default~\cite{maven:depend} (e.g., if dependencies for A, B, and C are defined as A -> B -> C -> D 2.0 and A -> E -> D 1.0, then D 1.0 will be used when building A because the path from A to D through E is shorter). User is able to manually exclude or force Maven to use a specific version, but the correctness is not guaranteed~\cite{maven:note}. Maven Enforcer Plugin checks a set of build rules including a Ban-Transitive-Dependency rule that detects all transitive dependencies conflicts and force users to resolve all conflicts. The Enforcer Plugin only checks the \codefont{groupid}, \codefont{artifactid}, and \codefont{version} declared in the configuration file (\todo{Lisa: though I don't know if we need to check semantics or compatibility of different versions, I leave this limitation for future reference}), and it does not help developers fix the conflicts without breaking any other parts of the system.
The majority of existing build systems model dependencies identical to Maven. Examples include Gradle, MSBuild, Make, and CloudMake. The features that make them different are unrelated to dependency management, but on syntax of build scripts and parallelization. 

Bazel is a new build system developed by Google that advertises parallelization and correctness~\cite{bazel:depend}. Bazel does not allow transitive dependency and only reads dependencies listed in from the root dependency declaration file. This means that if your project (A) depends on another project (B) which list a dependency on project C in its WORKSPACE file, both transitive dependencies from B and C should be added to the WORKSPACE file. This can balloon the file size, but hopefully limits the chances of having one library include C at version 1.0 and another include C at 2.0. But they also assume that users can provide correct configuration and select correct versions of libraries. 

Package managers in operating system also encounter similar dependency management problem. To deal with destructive upgrade and multi-distribution support in Linux products, Nix ~\cite{nix} provides a co-existing package management mechanism to store a source package in its own directory, instead of a global location that share across the entire system. However, this might not work in build system like Maven, as with transitive dependencies, the graph of included libraries can quickly grow quite large~\cite{maven:depend}. 

\section{Approach}

%There are some well-known solutions to this problem, they either ask developers to manually exclude all the transitive dependencies for a dependency (Solution 2) or include multiple versions in a side-by-side manner (Solution 3). Neither of them are perfect as described in the previous session. 

%We propose an approach to verify the safety of upgrading and 

%We propose an approach to check compatibility on the interfaces that use different versions for the same libraries, which has the possibility to introduce `dependency hell'. Instead of strict backward compatibility checker for the entire class~\cite{Welsch:backward12}, we regard it as safe if the used API is compatible for the other version in the current usage context. We make this trade-off because developers often embrace changes, making multiple versions incompatible to each other~\cite{bogart:backward}. To save build time and space,  we include multiple versions of the files from the same library only if they are incompatible with each other, instead of maintaining multiple versions for the same library at the same time. To save time for analysis, we only consider the API calls when it has the possibility to generate `dependency hell', since there is no need to check the dependency issue if the libraries are consistent used in the system with a single version.






\begin{figure}[!htb]
 \begin{minipage}{0.5\textwidth}
\scriptsize 
\begin{tabular}{@{}p{1\textwidth}} 
 \hline 
%  \multicolumn{1}{c}{(A) TextEditor} \\ \hline
  \vspace{-4mm}
\begin{Verbatim}[commandchars=\\\{\}, tabsize=2]
public class TextEditor extends JTextPane \{
\textcolor{purple}{ public UndoAction undoAction = new UndoAction();}
\textcolor{purple}{ public RedoAction redoAction = new RedoAction();}
\textcolor{purple}{ public CompoundUndoManager undo;}
 HashMap<Object, Action> actions = new HashMap<Object, Action>();
 public TextEditor(Workspace workspace) \{
   this.workspace = workspace;
\textcolor{purple}{   undo = new CompoundUndoManager(workspace);}
   actions.put("undo", undoAction);
   actions.put("redo", redoAction);
   \}
 public void discardUndoRedo() \{
    undo.discardAllEdits();
    undoAction.updateUndoState();
    redoAction.updateRedoState();
   \}
\textcolor{purple}{ public class UndoAction extends AbstractAction \{}
\textcolor{purple}{  public void actionPerformed(ActionEvent e) \{}
   try \{
\textcolor{purple}{     undo.undo();}
     updateUndoState();
     redoAction.updateRedoState();
    \} catch (CannotUndoException ex) \{
    \}\}
   public void updateUndoState() \{
    \uwave{setEnabled(undo.canUndo());}
   \}\}
\textcolor{purple}{ public class RedoAction extends AbstractAction \{}
\textcolor{purple}{   public void actionPerformed(ActionEvent e) \{}
    try \{
\textcolor{purple}{      undo.redo();}
      updateRedoState();
      undoAction.updateUndoState();
     \} catch (CannotRedoException ex) \{
     \}\}
      public void updateRedoState() \{
       \uwave{setEnabled(undo.canRedo());}
     \}\}
  \end{Verbatim}
      \vspace{-4mm}
     \\ \hline
\end{tabular} 
\caption{Result No 1: textmash:TextEditor}
\label{fig:textEditor}
\end{minipage}
\end{figure}





\begin{figure}[!htb]
 \begin{minipage}{0.5\textwidth}
\scriptsize 
\begin{tabular}{@{}p{1\textwidth}} 
 \hline 
%  \multicolumn{1}{c}{(B) No. 5.6,8, ConsoleTextEditor} \\ \hline
  \vspace{-4mm}
\begin{Verbatim}[commandchars=\\\{\}, tabsize=2]
public class ConsoleTextEditor extends JScrollPane \{
\textcolor{purple}{  private UndoAction undoAction = new UndoAction();}
\textcolor{purple}{    private RedoAction redoAction = new RedoAction();}
\textcolor{purple}{    private TextUndoManager undoManager;}
    public ConsoleTextEditor () \{
     this.undoManager = new TextUndoManager();
\textcolor{purple}{     doc.addUndoableEditListener(undoManager);}
     undoManager.addPropertyChangeListener(undoAction);
     undoManager.addPropertyChangeListener(redoAction);
\textcolor{purple}{     doc.addDocumentListener(undoAction);}
\textcolor{purple}{     doc.addDocumentListener(redoAction);}
  \}
\textcolor{purple}{ private class RedoAction extends UpdateCaretListener}
  implements PropertyChangeListener \{
\textcolor{purple}{  public void actionPerformed(ActionEvent ae) \{}
\textcolor{purple}{   undoManager.redo();}
   \uwave{setEnabled(undoManager.canRedo());}
   \uwave{undoAction.setEnabled(undoManager.canUndo());}
  \}
    public void propertyChange(PropertyChangeEvent pce) \{
     setEnabled(undoManager.canRedo());
   \} \}
\textcolor{purple}{ private class UndoAction extends UpdateCaretListener }
 implements PropertyChangeListener \{
\textcolor{purple}{   public void actionPerformed(ActionEvent ae) \{}
\textcolor{purple}{    undoManager.undo();}
    \uwave{setEnabled(undoManager.canUndo());}
    \uwave{redoAction.setEnabled(undoManager.canRedo());}
 \}
  public void propertyChange(PropertyChangeEvent pce) \{
    setEnabled(undoManager.canUndo());
  \}\}
  \end{Verbatim}
      \vspace{-4mm}
     \\ \hline
\end{tabular} 
\caption{Result No 5,6,8: groovy: DocumentUndoManagerImpl}
\label{fig:textEditor}
\end{minipage}
\end{figure}





%\section{Motivating Example}
\section{Usage Scenario}
\noindent{\textbf{Add undo and redo action for TextEditor.}}

\begin{table}[ht]
\begin{center}
\caption{Search Result from Code Search Engine}
\label{tab:total}
\scriptsize{
\begin{tabular*}{0.5\textwidth}{@{}c|llrr@{}} \hline
No.&Name&Project&LOC&\# M\\\hline
1&TextEditor&textmash&1270&12\\
2,3&AndroidTextEditor&android&576&9\\
4&DocumentUndoManagerImpl&ide&1227&17\\
5,6,8&ConsoleTextEditor&groovy&321&7\\ 
7&PapyrusCDTEditor&eclipse.papyrus&393&7\\
9,10&AspectEditorContributor&eclipse&88&2\\ \hline
\end{tabular*}
 \label{tab:undoResult}
 \textbf{ \# M} represents the number of methods that contain the query terms. 
 {\bf No.} represents the rank from CSE. Note that CSE may return the same results for multiple times, as the same file may exist in multiple branches. For instance, we regard 5,6,8 as identical with manual inspection. 
 
}
 \end{center}
\end{table}

To further illustrate that reuse task involves in multiple methods and classes, and the reuse task is hard without tool support, I imagine a scenario that user wants to add undo and redo actions for her Java Swing text editor application with undo/redo buttons, inspired by the evaluation task used in~\cite{Murphy:nlConcern11}. Without knowing any APIs, developer  first queries code search engine (CSE) with a free-form query \codefont{`undo redo TextEditor'}. Figure~\ref{fig:undoEditor} illustrates the implementation of `add undo/redo' concern mentioned in~\cite{Murphy:nlConcern11} and blue part represents source code that implements this feature.  Table~\ref{tab:undoResult} shows top 10 results returned from SearchCode CSE. We find that all code examples require multiple methods and classes to implement a feature. We also find that all code examples include other auxiliary features that are not directed related to the undo and redo feature. 

We assume that user provides a seed code example for the reuse task. She selects the first result from CSE. She uses keyword search to locate undo and redo feature in 6 methods and regards them as  seed API calls. Shown in Figure~\ref{fig:undoEditor}, she removes  the other  11 methods (865 LOC) that she regards as irrelevant to her reuse task. She recognizes that she needs to implement an \codefont{UndoAction} and \codefont{RedoAction} which are the subclass of \codefont{AbstractAction}, and overrides their actionPerform() method. In the actionPerformed() methods, she should invokes the \codefont{CompoundUndoManager.undo()} and  \codefont{CompoundUndoManager.redo()} correspondingly. Without knowing anything about \codefont{CompoundUndoManager}, she has to query CSE again for `CompoundUndoManager textmash'. This class consists of 17 methods (221 LOC) and she has to repeat to keyword search again to locate undo and redo feature in this class. With manual inspection, she notices that this \codefont{CompoundUndoManager} is a subclass of \codefont{javax.swing.UndoManager} and it overrides four methods \{\codefont{canUndo, canRedo, undo, redo}\} that are related to the feature. She notices that she actually does not need this \codefont{CompoundUndoManager} and decides to invoke its parent class \codefont{UndoManager}  instead. She also notices that she does not need the \codefont{CannotRedoException} by using \codefont{UndoManager}. 

After she integrates \codefont{UndoManager, UndoAction, and RedoAction} to her context, she tests it and it fails to perform the feature. She has to look at other examples, and she finds the No.5 example returned from CSE.  This example seems promising because it is also an implementation of a TextEditor. She notice that  the API \codefont{doc.addUndoableEditListener (undoManager)} seems the one that she misses now, but again, she has to investigate the class   \codefont{TextUndoManager} again. 

With this example, we illustrate that salient API calls that implement a desired feature are always interleaving with other related elements. To finish a reuse task, developer has to remove irrelevant parts.  However, it  requires significant effort to tease out these auxiliary features without tool support. Since the examples always contain both main feature and auxiliary features,  there seldom exists a perfect example for reuse and developers has to iteratively search for new examples and integrate it to their context until the integration results are examined.     

\section{Problem 3}
Problem: During copy/paste based reuse, developers must remove irrelevant parts and find salient API elements based on the context. 

Hypothesis: Our hypothesis is that having multiple examples of the same kind will help winnow out irrelevant parts, i.e, clustering multiple examples of the same kind and finding the commonality may help remove irrelevant parts. 

%We use the results for  the last three concerns implementation in Table~\ref{tab:concern} found by the code search engine SearchCode to illustrate the Problem 3. 



%  a feature location approach that is similar to~\cite{Denys:FCA12} to identify seed API calls shown in Figure~\ref{fig:undoEditor}. We remove the other  11 methods (865 LOC) that we believe is irrelevant to our reuse task. 

%For each partial program returned from CSE, we select the top k (k=5) methods that are related to the given query. I present No. 1, 5, 6, 8 to show that it requires multiple methods and classes to implement a concern. The other four examples are shown in Appendix.

%In this example, we use feature location approach~\cite{Denys:FCA12} to identify salient API calls, 


















\begin{comment}
\noindent{\textbf{Example collection phase.}} With 7  partial programs at hand, it first extracts class-level structural facts for each code example. Based on a relative match threshold to measure the relative occurrence of the number of matched facts out of all class-level facts,  our tool clusters these examples in three groups using complete linkage technique: 5 examples use classes in Java Swing like `AbstractAction' and `UndoableEditListener', another example uses a `Node' data structure to illustrate how to implement undo/redo actions using command design pattern, and the last example is the javadoc for `UndoManager'.  Our tool starts from the first cluster as it has the highest relative occurrence out of all examples. Based on partial program analysis, we generate a list of type facts to infer types and bind methods.

The common facts our tool extracts include: 
\begin{table}[ht]
\begin{center}
\caption{Common Structural Facts Extracted from the Examples }
\label{tab:total}
\vspace{1mm}
\scriptsize{
\begin{tabular*}{0.5\textwidth}{@{}l|l@{}} \hline
Level&Structural Facts\\\hline
class&subType(Undo*Action, AbstractAction), \\ 
&subType(Redo*Action, AbstractAction), \\
&subType(*Listener, UndoListener)\\ \hline
method&override(actionPerformed(), Undo*Action, AbstractAction), \\
&override(actionPerformed(), Redo*Action, AbstractAction), \\ \hline
statement&init(*,javax.swing.undo.UndoManager), \\
&invoke(UndoManager.undo, actionPerformed(), Undo*Action),\\
& invoke(UndoManager.redo, actionPerformed(), Redo*Action)\\  \hline
\end{tabular*}
 \label{tab:dataset}

We simplify some full names due to space limitation: 

AbstractAction: javax.swing.AbstractAction;

UndoListener: javax.swing.event.UndoableEditListener

actionPerformed():actionPerformed(ActionEvent)
}
 \end{center}
\end{table}



\noindent{\textbf{Example clustering phase.}} After extracting facts for each example, our tool identifies main facts shared in all 5 examples shown in Figure~\ref{fig:fact} part A. It leverages slicing to group related facts. In this example, 3 of them add \codefont{Undo*Action} to \codefont{JMenuItem} by invoking \codefont{JMenuItem.add (ActionEvent)} and support undo and redo actions for Text Editor shown in Figure~\ref{fig:fact} part B. The other two of them invoke \codefont{JButton.addActionListener(ActionEvent)} to add these two actions to \codefont{JButton}, and support \codefont{JApplet} Shape drawing shown in Figure~\ref{fig:fact} part C. After extracting both main facts and related facts, we generate two reuse tasks by combining main facts with each related fact group.  

\noindent{\textbf{Example integration phase.}} Given a context like Figure~\ref{fig:context} part D, our tool identifies the best-fit reuse task based on the number of matched facts in the user's context.  If our tool fails to identify any matched facts in the user's context, it will simply rank the reuse task based on its occurrence in the code examples. Our tool asks for user's selection by presenting the generated main facts and related facts to the user in the form of code snippet. 
After confirmed by users, our tool automatically fills the rest facts that do not exist in the context shown in Figure~\ref{fig:context} part E. 
\end{comment}


\begin{figure*}[!htb]
    \begin{minipage}{0.45\textwidth}
    \centering
\includegraphics[width=1\textwidth]{fig/clipboard.jpeg}
       \caption{Drag-and-drop example for systematic editing}
        \label{fig:clipboard}
 \end{minipage}%
  \hfill
     \begin{minipage}{0.45\textwidth}
     \centering
 \includegraphics[width=1\textwidth]{fig/clipboard-grid.jpeg}
       \caption{Example for Drag-and-drop}
        \label{fig:dndweb}
%  \includegraphics[width=1\textwidth]{fig/select.jpeg}
%       \caption{Example for selection}
%        \label{fig:selectWeb}       
  \end{minipage}
    \hfill
   \end{figure*}
   
   \begin{comment}
\begin{figure}[!htb]
 \begin{minipage}{0.5\textwidth}
\scriptsize 
\begin{tabular}{@{}p{1\textwidth}} 
 \hline 
  \multicolumn{1}{c}{(A) Common facts} \\ \hline
  \vspace{-4mm}
\begin{Verbatim}[commandchars=\\\{\}, tabsize=2]
import javax.swing.undo.UndoManager;
import javax.swing.event.UndoableEditListener;
import javax.swing.AbstractAction;

public class Foo \{
  public void run() \{
   UndoManager undoManager = new UndoManager();
  \}\}
  class UndoListener implements UndoableEditListener \{
  public void undoableEditHappened(UndoableEditEvent e) \{
    undoManager.addEdit(e.getEdit());
  \}\}
class UndoAction extends AbstractAction \{
   public void actionPerformed(ActionEvent e) \{   
     undoManager.undo();  
  \}\}
class RedoAction extends AbstractAction \{
  public void actionPerformed(ActionEvent e) \{
     undoManager.redo();
  \}\}
\end{Verbatim}
\vspace{-4mm}
 \\ \hline
  \multicolumn{1}{c}{(B) Auxiliary facts: Menu} \\ \hline
    \vspace{-4mm}
\begin{Verbatim}[commandchars=\\\{\}, tabsize=2]
import com.sun.java.swing.*;

public class Foo \{
  public void run() \{
   UndoManager undoManager = new UndoManager();
   new TextComposite().getDocument().addUndoableEditListener(new UndoListener());
   new JMenuItem(new UndoAction());
   new JMenuItem(new RedoAction());
  \}\}
 \end{Verbatim}
   \vspace{-4mm}
  \\ \hline
   \multicolumn{1}{c}{(C) Auxiliary facts: Button} \\ \hline
  \vspace{-4mm}
\begin{Verbatim}[commandchars=\\\{\}, tabsize=2]
import javax.swing.undo.UndoableEditSupport;
import com.sun.java.swing.Button;

public class Foo \{
  public void run() \{
   UndoManager undoManager = new UndoManager();
   new Button.addActionListener(new UndoAction());
   new Button.addActionListener(new RedoAction());
   new UndoableEditSupport().addUndoableEditListener(new UndoListener());
  \}\}
  \end{Verbatim}
        \vspace{-4mm}
     \\ \hline
\end{tabular} 
\caption{Common features and auxiliary features extracted from informal resources}
\label{fig:fact}
\end{minipage}
\end{figure}
\end{comment}


\begin{figure}[!htb]
 \begin{minipage}{0.5\textwidth}
\scriptsize 
\begin{tabular}{@{}p{1\textwidth}} 
 \hline 
  \multicolumn{1}{c}{(A) User's context} \\ \hline
  \vspace{-4mm}
\begin{Verbatim}[commandchars=\\\{\}, tabsize=2]
import com.sun.java.swing.*;

public class MyTextEditor \{
  public void init() \{
    JFrame frame = new JFrame("Undo Sample");
    frame.setDefaultCloseOperation(JFrame.EXIT_ON_CLOSE);
    JTextArea textArea = new JTextArea();
   JButton undoBtn_;
   //add undo and redo action to text editor
  \}\} 
\end{Verbatim}
    \vspace{-4mm}
     \\ \hline
  \multicolumn{1}{c}{(E) Expected reuse plan based on the query `undo redo TextEditor'} \\ \hline
\begin{Verbatim}[commandchars=\\\{\}, tabsize=2]
import com.sun.java.swing.*;
import javax.swing.undo.UndoManager;
import javax.swing.event.UndoableEditListener;
import javax.swing.AbstractAction;

public class MyTextEditor \{
    JFrame frame = new JFrame("Undo Sample");
    frame.setDefaultCloseOperation(JFrame.EXIT_ON_CLOSE);
    JTextArea textArea = new JTextArea();
    JButton undoBtn_;
\textcolor{blue}{    JButton redoBtn_; }
  
\textcolor{blue}{   public void init() \{}
   //add undo and redo action to text editor
\textcolor{blue}{   UndoManager undoManager = new UndoManager(); }
\textcolor{blue}{   textArea.getDocument().addUndoableEditListener(new UndoListener());}
\textcolor{blue}{   undoBtn_ = new Button();}
\textcolor{blue}{   undoBtn_.addActionListener(new UndoAction());}
\textcolor{blue}{   redoBtn_ = new Button().addActionListener(new RedoAction());}
\textcolor{blue}{   redoBtn_.addActionListener(new RedoAction());}
\textcolor{blue}{   \}}
\textcolor{blue}{ private class UndoListener implements UndoableEditListener \{}
\textcolor{blue}{  public void undoableEditHappened(UndoableEditEvent e) \{}
\textcolor{blue}{    undoManager.addEdit(e.getEdit());}
\textcolor{blue}{  \}\}}
\textcolor{blue}{private class UndoAction extends AbstractAction \{}
\textcolor{blue}{  public void actionPerformed(ActionEvent e) \{   }
\textcolor{blue}{    undoManager.undo(); }
\textcolor{blue}{  \}\}}
\textcolor{blue}{private class RedoAction extends AbstractAction \{}
\textcolor{blue}{  public void actionPerformed(ActionEvent e) \{}
\textcolor{blue}{     undoManager.redo();}
\textcolor{blue}{  \}\}\}}
  \end{Verbatim}
      \vspace{-4mm}
     \\ \hline
\end{tabular} 
\caption{Scenario 1: add undo/redo to a TextEditor}
\label{fig:undoEditor}
\end{minipage}
\end{figure}






   


% yet developers still have to manually go through all related elements and rely on their intuition to determine where to look.  


%balance two competing concerns: the desire to reuse as much code as possible to obtain the needed functionality of their task, and the desire to eliminate as much irrelated reused code as practical.  save time and to increase the quality of their code


%Gilligan~\cite{Holmes:reuse07} and Procrustes~\cite{Holmes:ASE09} suggest to reuse program elements based on structural relevance and cost of reuse, yet developers still have to manually go through all related elements and rely on their intuition to determine where to look.  

%developers have to select the `best' example that provides desired functionality for the task and integrate well to the target context; second, even though the user is able to select the      

%Gilligan~\cite{Holmes:reuse07} and Procrustes~\cite{Holmes:ASE09} try to address this problem in the context of large-scale reuse tasks by suggesting program elements that are easy to  based on structural relevance and cost of reuse, while Jigsaw~\cite{Cottrell:jigsaw08} supports small-scale integration of source code into target system by considering structural and semantic similarity measures between the example and target context. However, in any non-trivial software system, the number of strut yet these approaches present information in terms of relevance to the user query thus the choice of examples and the order does not consider the developer's need to find an example that integrates well. The next step to identify if the problem can be This process is not trivial: first, in any non  investigate these examples to identify those that match the task and can be integrate to the target context,  developers still have to investigate related elements for each example manually and integrate 

%Reuse tasks can be divided into three phases: {\em location}, in which developers queries for a set of examples considering the task; {\em selection}, in which users investigate these examples to identify those that match the task and can be integrate to the target context;  and {\em integration}, in which developers copies and modifies the chosen example to fit for the target context~\cite{Holmes:reuseStudy09}. These phases can be iterative as there is no guarantee that a selected example will be appropriate until the integration results are examined. When exploring the source code they want to reuse, developers balance two competing concerns: the desire to reuse as much code as possible to obtain the needed functionality of their task, and the desire to eliminate as much irrelated reused code as practical to save maintenance cost. an API by investigating examples of its use. perform reuse tasks to obtain the needed functionality of their task efficiently~\cite{Frakes:reuseStatus05, Parsons:cognitiveReuse04}.  integrating existing usage within target system.  



%Reuse tasks can be divided into two phases: location, in which developers queries for a set of examples considering the task; and integration, in which developers copies and modifies the chosen example to fit for the target context. Both phases can be iterative as there is no guarantee that a selected examples will be appropriate until the integration results are examined.  It is common to start a reuse task by finding related examples, investigating these examples to consider its related functionality for the task, eliminating spurious code that will increase maintenance costs, and modifying the chosen example to fit for the target context. Although some approaches advocate refactoring code rather than reuse code in this manner, recent researches have found that these kind of clone cannot be easily refactored and have to be modified to meeting requirements in new context. Numerous studies have shown the necessity of a so-called `white box' reuse that modify existing example to fit for the new context: Frake et al.\/note that most software systems are variant on existing ones and reuse strategies within companies. Serly shows that reused code analyzes 25 projects at NASA and finds that 32\% of modules within these were reused from prior projects, of these reused modules, 47\%  Parsons et al.\/present that developers always anchor their understanding to existing code and adjust code to meet their needs. locating source code examples based on user query and return When exploring the source code they want to reuse, developers balance two competing concerns: the desire to reuse as much code as possible to obtain the needed functionality of their task, and the desire to eliminate as much irrelated reused code as practical.  save time and to increase the quality of their code. Numerous approaches exist to help developers locate source code examples, in which the developers queries for a set of examples to consider for relevance to the task; some helps developers investigates those examples. Reuse in this manner can be seen as crating code clones. While these clones have in the past been perceived negatively, recent research has found that there are a large amount of clones that are not easily refactored. non-refactorable clones 










%Software maintainers spend a lot of time trying to understand the existing software before performing the actual modification. For modification tasks such as debugging, finding and understanding program elements related to a bug fix tends to take more time than actually fixing the bug.  Finding related code elements  is used in a variety of software development and maintenance activities. For instance, defect prediction tools identify fault-prone elements via dependency graphs \cite{Nagappan:ICSE06, Zimmermann:ICSE08, PRMiner:FSE05}, impact analysis tools estimate potentially impacted entities by analyzing related elements of a proposed change  \cite{Denys:impactMetrics13, Orso:impactDynamic03}, API search tools identify related elements in the call graph based on user query  \cite{Denys: portfolio11, sniff:Sen09}, and feature localization tools find related elements which represent a specific concern  \cite{sniafl: TSE06, Hassan:ICSE10, Hill:locateConcern07}.  


%Numerous measures have been proposed to identify related program elements by leveraging execution trace~\cite{Briand:dynamicTSE04}, historical changes~\cite{Ying:cochangeTSE04, Gall:changeCouple08}, and lexical information in identifiers and documents~\cite{Denys:couple11, Hill:neighbor07}, yet these data may not always be available and reliable compared to source code.  A study on program investigation \cite{Robillard:TSE04} shows that effective developers tend to find related elements by following structural dependencies. However, in any non-trivial software system, the number of structural dependencies to follow is much too large to be completely covered by a developer. As a result, developers must rely on their intuition to determine where to look. Based on the hypothesis that {\em it is possible to find  related elements following program's structural dependency,}  a variety of approaches have been reported to suggest elements of potential interest via structural metrics \cite{Briand:structural99}, program slicing \cite{Bodik:slicePLSI07}, and topology-based analysis~\cite{Robillard:FSE05, Zimmermann:ICSE08}.  We observe several limitations for existing approaches: First, we find that most techniques only support a single queried elements and work at a fixed granularity in a limited scope, e.g., direct callers/callees or sibling elements that share callers/callees, but they do not have a good representation for relations in different level, such as separating overriding relations across different classes and calling relations within a single method.  Second, existing tools are sensitive to specific callers/callees but do not consider how queried elements are related with them  in a common manner, i.e., existing tools do not consider which relations are more common and should be more interesting to users. Third, existing tools assume that it takes equal effort to explore each relation, yet recent study shows that elements with many nearby dependencies require more effort for investigation~\cite{Holmes:ASE09} and hierarchical relations are easier to follow and may have a higher potential to find interesting elements than calling relations~\cite{Murphy:nlConcern11}. 

%To overcome these limitations, we propose to merge and cluster similar couplings in a layered-order given a set of program elements queried by users, and return a set of grouped elements that are related to the given set.  We hypothesize that when users query one or more elements, they are  interested in not only which elements are directly related to these elements, but also how these elements are coupled with other related elements in a common way, and these commonly related elements should be prioritized.  


%For example, user selects two bolded methods in class \codefont{JsonWriter}:  \codefont{visit(JsonNumber)} and \codefont{visit(JsonBoolean)}. The underline elements represent methods that are invoked in queried ones: \codefont{print()} method, \codefont{JsonNumber.value}, and \codefont{JsonBoolean.value}. Suade~\cite{Robillard:FSE05}, the only approach that supports multiple elements of interest, reports these three invoked methods, as well as other 13 callers of these two methods based on class hierarchy analysis to include all methods potentially called. These results can be useful, but a) Suade fails to detect any direct relations between two queried methods and investigates direct related elements for them separately, and report invoked methods (\codefont{JsonNumber.value()} and \codefont{JsonBoolean.value()}) and callers (\codefont{JsonNumber.visit()} and \codefont{JsonBoolean.visit()}) separately; b) the results are sensitive to the callers of queried elements (\codefont{JsonNumberOp- tional.recordOptional()}, \codefont{JsonFormatNumberHandler.print Number()}, \codefont{JSON.write()}, and the rest 10 callers). I hypothesize that related elements that users are interested in should be \codefont{<JsonValue>.visit()} and \codefont{JSON.write()} as callers, and \codefont{<JsonValue>.value()}  and \codefont{print()} as methods invoked by queried elements. 


%We state our research question as: {\em Given a set of program elements defined by users, which elements are related to the set of interest with respect to a relation vocabulary?}
% and code reuse \cite{Holmes:ASE09, Holmes:structural05}
%\begin{table}[ht]
\begin{center}
\caption{Comparison of representative code search tools, code reuse tools, and code transformation tools}
\label{tab:total}
\vspace{1mm}
\scriptsize{
\begin{tabular*}{0.5\textwidth}{@{}l|ccccccc@{}} \hline
Tool&I&$O_{m}$&Related&Cluster&Partial&Transfer\\\hline
RecoDoc~\cite{RecoDoc:ICSE12}&D&&&&\checkmark\\
ACE~\cite{PeterACE:ICSE13}&D&&&&\checkmark\\
MAPO~\cite{MAPO:ECOOP09}&M&&&\checkmark&\checkmark\\
Buse et al.~\cite{Buse:apiICSE12}&M&&&\checkmark&\checkmark\\
SNIFF~\cite{sniff:Sen09}&Q&&&\checkmark&\checkmark\\
%Strathcona~\cite{Holmes:structural05}&C&C&&\checkmark&&\\
%Prompter~\cite{Prompter:MSR14}&W&C&&\checkmark&&\\ 
%Portfolio~\cite{Portfolio:DenysICSE11}&W&Q&&\checkmark&&\\
%Export~\cite{Export:DenysASE13}&C&M&&\checkmark&&\\
%Sourcerer~\cite{Sourcerer:SC14}&C&Q&&\checkmark&&\\ 
%CodeGenie~\cite{TDCS:SAC09}&C&Q&&\checkmark&&\\

%Prospector~\cite{Mandelin:jungloid05}&C&T&&\checkmark\\ 
%PRIME~\cite{Prime:OOPSLA12}&C&Q&&\checkmark&&\\
MUSE~\cite{MUSE:MarcusICSE15}&M&&\checkmark&\checkmark&\\ 
Keivanloo~\cite{spotWork:ICSE14}&Q&&\checkmark&\checkmark&\\
PRIME~\cite{Prime:OOPSLA12}&Q&&\checkmark&&\checkmark\\
%\hline
%SSI~\cite{SSI:FSE10}&C&Q&&\checkmark&&\\
%PARSEWeb~\cite{parseWeb:ASE07}&\\
%CodeHint~\cite{CodeHint:ICSE14}&\\  \hline
Gilligan~\cite{Holmes:reuse07}&E&\checkmark&\checkmark&&\\
%Procrustes~\cite{Holmes:ASE09}&E&\checkmark&\checkmark&&\\ 
Jigsaw~\cite{Cottrell:jigsaw08}&E&&\checkmark&&&\checkmark\\ 
LASE~\cite{LASE:ICSE13}&E&&\checkmark&\checkmark&&\checkmark\\ \hline
%Grapacc~\cite{Grapacc:ICSE12}&C&E&&&&\checkmark\\ \hline
Ours&Q&\checkmark&\checkmark&\checkmark&\checkmark&\checkmark\\
\end{tabular*}
 \label{tab:dataset}
\vspace{0.1cm}

%The {\bf Source} column specifies the input source ({\bf C}orpus, {\bf W}eb, {\bf U}ser);
The {\bf I} column specifies the input  format (Informal {\bf D}ocument, {\bf M}ethod, free-form {\bf Q}uery, {\bf E}xample); 
The {\bf $O_{m}$} column specifies if the  output is across the method boundary; The {\bf Related} column specifies if the tool considers related methods when recommending example or performing transformation; 
The {\bf Cluster} column specifies if the output results are clustered to reduce redundancy; 
The {\bf Partial} column specifies if the tool supports partial program as input.
The {\bf Transfer} column specifies if the tool is able to transfer the code to another context.
}
 \end{center}
\end{table}


%MAPO:ECOOP09, Buse:apiICSE12, sniff:Sen09, Holmes:structural05, Prompter:MSR14, Portfolio:DenysICSE11, Export:DenysASE13, Prime:OOPSLA12, MUSE:MarcusICSE15, spotWork:ICSE14





%\section{Related Work}


%Most measures for related element identification are {\em structural} by following method call chains, control flow, and variable def-use. These works leverage coupling metrics~\cite{Briand:structural99}, program slicing \cite{Bodik:slicePLSI07}, and topology-based search~\cite{Robillard:FSE05, Zimmermann:ICSE08}. 


%\noindent{\bf{Static Metrics.}} Briand et al.\/ propose a set of structural coupling metrics to measure coupling between two classes. such as Coupling Between Objects (CBO) and Response for a Class (RFC). A set of metrics are proposed to predict defects based on the complexity or defect history~\cite{Gyimothy:metrics05, Nagappan:ICSE06}. Yet static metrics are usually too coarse-grained that limits their usefulness to recommend code of immediate interest.

%\noindent{\bf{Program slicing.}} Weiser~\cite{Weiser:slice84} proposes to identify  a part of program that may affect the values computed at some point of interest. Thin slicing~\cite{Bodik:slicePLSI07} only consists of producer statements for the seed and hierarchically expanded to include statements explaining how producers affect the seed. Yet computing slicing can be expensive and slices are often too large to be useful for users. 


%\noindent{\bf{Topology-based analysis.}} Suade algorithm \cite{Robillard:FSE05}  ranks elements based on the closeness of their structural association with program elements in a set of interest. It only returns elements that are directly related to the set of interest. FRAN~\cite{Devanbu:randomWalk07} extends Suade by considering the sibling elements that share common caller or callee with queried elements. FRAN focuses on calling relation in C given a single queried function. Zimmermann and Nagappan \cite{Zimmermann:ICSE08} use network analysis to identify  {\em central binaries} based on dependency graph, aiming to predict fault-prone modules based on defect history. Other usage of identifying related elements rely on calling graphs with class hierarchy analysis for dynamic binding \cite{PRMiner:FSE05, Orso:impactDynamic03, Murphy:nlConcern11}. However, none of them groups relations in different granularities and identifies commonly used relations in layered-order. 
 
%Considering that structural coupling often results in a large number of relationships within a limited scope, researchers have proposed other alternative measures to identify related elements.

%Briand et al.~\cite{Briand:dynamicTSE04} handle dynamic binding in object-oriented programs by analyzing of the execution of a program. Dynamic slicing is a variant of slicing that select related code piece considering program execution trace \cite{Agrawal:dynamicSlice90}. Specifically, dynamic slicing only considers program dependencies that occur in a specific execution of the program. In contrast to static approaches, dynamic measures relies on the availability and quality of test cases for an executable program, thus they cannot be applied to incomplete code or code that cannot be executed. 

 %Ying et al.\cite{Ying:cochangeTSE04} report elements that are often changed together during program evolution tasks at the file level, Fluri et al.\/ identify frequently co-changed files and filter co-changed methods that result from structural changes \cite{Gall:changeCouple05}. However, reliance on change history implies that the approach cannot be used when queried elements are never changed before. 


%Revelle et al.~\cite{Denys:couple11} exploit relations captured from the source code lexicon using Information Retrieval techniques and propose a set of semantic metrics to define a coupling between classes based on textural information from code and comments. Hill et al.~\cite{Hill:neighbor07} use lexical information to identify related methods  corresponding to queried ones and expand call chains to find related elements missed by lexical search.  The main tradeoff for these measures is that they assume that similar terms always indicate related functionality.  

%In summary, these alternative measures  aim to capture relations that are missed by existing structural coupling by leveraging revision history, lexical information, and execution traces. Our approach aims to extend the structural measures to group relations in different granularities and identify commonly used relations based on available and reliable structural dependency.  



%\bibliographystyle{abbrvnat}
% \renewcommand{\bibfont}{\footnotesize} % <--- change bib font size here
% \setlength{\bibsep}{0.5ex}             % <--- change space between bib entries here
\bibliographystyle{abbrv}
\bibliography{coupling,reuse} 
%\bibliography{strings-short,paper}  % <--- use short strings in case of emergency

\end{document}
