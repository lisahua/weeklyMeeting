\begin{figure}[!htb]
 \begin{minipage}{0.47\textwidth}
\scriptsize 
\begin{tabular}{@{}p{1\textwidth}} 
 \hline 
  \multicolumn{1}{c}{(A) Drag-and-Drop View in the Table Format } \\ \hline
  \vspace{-4mm}
\begin{Verbatim}[commandchars=\\\{\}, tabsize=2]
public class TableViewDisplay extends Display \{
@override
  public void run(Composite parent) \{
    TableViewer table = new Table(parent, SWT.BORDER | 
      SWT.H_SCROLL | SWT.V_SCROLL);
    GridLayout layout = new GridLayout();
    layout.numColumns = 2;
    layout.makeColumnsEqualWidth = true;
    table.setLayout(layout);
    Transfer[] types = new Transfer[] \{TextTransfer.getInstance()\};
    int dnd = DND.DROP_MOVE | DND.DROP_COPY;
    table.setTransfer(types);
    table.setOperation(dnd);
    table.addDragSupport(new DragListener());
    table.addDropSupport(new DropListener(table));
  \} \}
\end{Verbatim}
\vspace{-4mm}
 \\ \hline
  \multicolumn{1}{c}{(B) DropListener that is related to the main functionality} \\ \hline
    \vspace{-4mm}
\begin{Verbatim}[commandchars=\\\{\}, tabsize=2]
public class DropListener implements DropTargetListener \{
    private TableViewer table;
    public DropListener(TableView table) 
       \{ this.table = table; \}
 @Override
    public void drop(DropTargetEvent event) \{
      TableItem item = (TableItem) event.data;
       table.add(item);
   \} \}
 \end{Verbatim}
   \vspace{-4mm}
  \\ \hline
   \multicolumn{1}{c}{(C) DragListener  that is related to the main functionality} \\ \hline
  \vspace{-4mm}
\begin{Verbatim}[commandchars=\\\{\}, tabsize=2]
public class DragListener implements DragSourceListener \{
   @Override
  public void dragSetData(DragSourceEvent event) \{
    TableItem item = (TableItem) event.getSelection();
    event.data = item.toString();
  \}
\end{Verbatim}
\vspace{-4mm}
 \\ \hline
\end{tabular} 
\caption{Drag-and-Drop Example for shopping cart}
\label{fig:cartTable}
\end{minipage}
\end{figure}


\begin{figure}[!htb]
 \begin{minipage}{0.47\textwidth}
\scriptsize 
\begin{tabular}{@{}p{1\textwidth}} 
 \hline 
  \multicolumn{1}{c}{(A) Drag-and-Drop View in Eclipse Viewer } \\ \hline
  \vspace{-4mm}
\begin{Verbatim}[commandchars=\\\{\}, tabsize=2]
public class ListViewPart extends ViewPart \{
  @Override
  public void createPartControl(Composite parent) \{
    ListViewer viewer = new ListViewer(parent, 
       SWT.H_SCROLL|SWT.V_SCROLL);
    int ops = DND.DROP_COPY | DND.DROP_MOVE;
    Transfer[] transT = new Transfer[]\{TextTransfer.getInstance()\};
    viewer.setTransfer(transT);
    viewer.setOperation(ops);
    viewer.addDragSupport(new DragListener());
    viewer.addDropSupport(new DropListener(viewer));
    viewer.setContentProvider(new TodoModelProvider());
  \} \}
\end{Verbatim}
\vspace{-4mm}
 \\ \hline
  \multicolumn{1}{c}{(B) DropListener that is related to the main functionality } \\ \hline
    \vspace{-4mm}
\begin{Verbatim}[commandchars=\\\{\}, tabsize=2]
public class DropListener implements DropTargetListener \{
  private ListViewer viewer;
  public DragListener() 
  \{ this.viewer = viewer; \}
 @Override
    public void drop(DropTargetEvent event) \{
        ISelection sel=(ISelection) event.data;
        viewer.add(sel.toString());
   \} \}
 \end{Verbatim}
   \vspace{-4mm}
  \\ \hline
   \multicolumn{1}{c}{(C) DragListener that is related to the main functionality} \\ \hline
  \vspace{-4mm}
\begin{Verbatim}[commandchars=\\\{\}, tabsize=2]
public class DragListener implements DragSourceListener \{
   @Override
  public void dragSetData(DragSourceEvent event) \{
    ISelection sel = event.getSelection();
    event.data = sel.toString();
  \}
\end{Verbatim}
\vspace{-4mm}
 \\ \hline
    \multicolumn{1}{c}{(D) Structured Data Provider for Drag-and-Drop} \\ \hline
  \vspace{-4mm}
\begin{Verbatim}[commandchars=\\\{\}, tabsize=2]
public class TodoModelProvider implements IStructuredContentProvider \{
  @override
    public Object[] getElements(Object inputElement) \{
    List<Todo> list = (List<Todo>) inputElement;
    return list.toArray();
  \}
  \end{Verbatim}
\vspace{-4mm}
 \\ \hline
\end{tabular} 
\caption{Drag-and-Drop Example for TODO labels}
\label{fig:todoList}
\end{minipage}
\end{figure}



%%%%%%%Usage pattern example %%%%%%%

%
%\begin{figure}[!htb]
% \begin{minipage}{0.47\textwidth}
%\scriptsize 
%\begin{tabular}{@{}p{1\textwidth}} 
% \hline 
% \multicolumn{1}{c}{(A) An inconsistent object name found by both Lancelot and \tool } \\ \hline
%  \vspace{-4mm}
%\begin{Verbatim}[commandchars=\\\{\}, tabsize=2]
% /* org.elasticsearch.action.delete.DeleteRepositoryRequestBuilder */
%1.public String setName() \{
%2.  return setName;
%3.\}
%\end{Verbatim}
%\vspace{-4mm}
% \\ \hline
%  \multicolumn{1}{c}{(B) A naming bug from Lancelot} \\ \hline
%   \vspace{-4mm}
%\begin{Verbatim}[commandchars=\\\{\}, tabsize=2]
%/* org.elasticsearch.action.termvector.TermVectorFields */
%1.public void size(int size) \{
%2.  if (size <= 0) 
%3.   throw new legalArgumentException("Size must be positive"); 
%4.  this.size = size;
%5.\}
% \end{Verbatim}
% Lancelot said: Methods with this name never return void.
% \vspace{-4mm}
%  \\ \hline
%\end{tabular} 
%\caption{Name Checker comparison}
%\label{fig:compare}
%\end{minipage}
%\end{figure}
 