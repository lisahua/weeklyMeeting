\begin{comment}
\begin{figure}[!htb]
 \begin{minipage}{0.5\textwidth}
\scriptsize 
\begin{tabular}{@{}p{1\textwidth}} 
 \hline 
  \multicolumn{1}{c}{(A) Common facts} \\ \hline
  \vspace{-4mm}
\begin{Verbatim}[commandchars=\\\{\}, tabsize=2]
import javax.swing.undo.UndoManager;
import javax.swing.event.UndoableEditListener;
import javax.swing.AbstractAction;

public class Foo \{
  public void run() \{
   UndoManager undoManager = new UndoManager();
  \}\}
  class UndoListener implements UndoableEditListener \{
  public void undoableEditHappened(UndoableEditEvent e) \{
    undoManager.addEdit(e.getEdit());
  \}\}
class UndoAction extends AbstractAction \{
   public void actionPerformed(ActionEvent e) \{   
     undoManager.undo();  
  \}\}
class RedoAction extends AbstractAction \{
  public void actionPerformed(ActionEvent e) \{
     undoManager.redo();
  \}\}
\end{Verbatim}
\vspace{-4mm}
 \\ \hline
  \multicolumn{1}{c}{(B) Auxiliary facts: Menu} \\ \hline
    \vspace{-4mm}
\begin{Verbatim}[commandchars=\\\{\}, tabsize=2]
import com.sun.java.swing.*;

public class Foo \{
  public void run() \{
   UndoManager undoManager = new UndoManager();
   new TextComposite().getDocument().addUndoableEditListener(new UndoListener());
   new JMenuItem(new UndoAction());
   new JMenuItem(new RedoAction());
  \}\}
 \end{Verbatim}
   \vspace{-4mm}
  \\ \hline
   \multicolumn{1}{c}{(C) Auxiliary facts: Button} \\ \hline
  \vspace{-4mm}
\begin{Verbatim}[commandchars=\\\{\}, tabsize=2]
import javax.swing.undo.UndoableEditSupport;
import com.sun.java.swing.Button;

public class Foo \{
  public void run() \{
   UndoManager undoManager = new UndoManager();
   new Button.addActionListener(new UndoAction());
   new Button.addActionListener(new RedoAction());
   new UndoableEditSupport().addUndoableEditListener(new UndoListener());
  \}\}
  \end{Verbatim}
        \vspace{-4mm}
     \\ \hline
\end{tabular} 
\caption{Common features and auxiliary features extracted from informal resources}
\label{fig:fact}
\end{minipage}
\end{figure}
\end{comment}















\begin{figure}[!htb]
 \begin{minipage}{0.5\textwidth}
\scriptsize 
\begin{tabular}{@{}p{1\textwidth}} 
 \hline 
  \multicolumn{1}{c}{(A) User's context} \\ \hline
  \vspace{-4mm}
\begin{Verbatim}[commandchars=\\\{\}, tabsize=2]
import com.sun.java.swing.*;

public class MyTextEditor \{
  public void init() \{
    JFrame frame = new JFrame("Undo Sample");
    frame.setDefaultCloseOperation(JFrame.EXIT_ON_CLOSE);
    JTextArea textArea = new JTextArea();
   JButton undoBtn_;
   //add undo and redo action to text editor
  \}\} 
\end{Verbatim}
    \vspace{-4mm}
     \\ \hline
  \multicolumn{1}{c}{(E) Suggested reuse plan based on user's context} \\ \hline
\begin{Verbatim}[commandchars=\\\{\}, tabsize=2]
import com.sun.java.swing.*;
import javax.swing.undo.UndoManager;
import javax.swing.event.UndoableEditListener;
import javax.swing.AbstractAction;

public class MyTextEditor \{
    JFrame frame = new JFrame("Undo Sample");
    frame.setDefaultCloseOperation(JFrame.EXIT_ON_CLOSE);
    JTextArea textArea = new JTextArea();
    JButton undoBtn_;
\textcolor{blue}{    JButton redoBtn_; }
  
\textcolor{blue}{   public void init() \{}
   //add undo and redo action to text editor
\textcolor{blue}{   UndoManager undoManager = new UndoManager(); }
\textcolor{blue}{   textArea.getDocument().addUndoableEditListener(new UndoListener());}
\textcolor{blue}{   undoBtn_ = new Button();}
\textcolor{blue}{   undoBtn_.addActionListener(new UndoAction());}
\textcolor{blue}{   redoBtn_ = new Button().addActionListener(new RedoAction());}
\textcolor{blue}{   redoBtn_.addActionListener(new RedoAction());}
\textcolor{blue}{   \}}
\textcolor{blue}{ private class UndoListener implements UndoableEditListener \{}
\textcolor{blue}{  public void undoableEditHappened(UndoableEditEvent e) \{}
\textcolor{blue}{    undoManager.addEdit(e.getEdit());}
\textcolor{blue}{  \}\}}
\textcolor{blue}{private class UndoAction extends AbstractAction \{}
\textcolor{blue}{  public void actionPerformed(ActionEvent e) \{   }
\textcolor{blue}{    undoManager.undo(); }
\textcolor{blue}{  \}\}}
\textcolor{blue}{private class RedoAction extends AbstractAction \{}
\textcolor{blue}{  public void actionPerformed(ActionEvent e) \{}
\textcolor{blue}{     undoManager.redo();}
\textcolor{blue}{  \}\}\}}
  \end{Verbatim}
      \vspace{-4mm}
     \\ \hline
\end{tabular} 
\caption{Scenario 1: add undo/redo to a TextEditor}
\label{fig:undoEditor}
\end{minipage}
\end{figure}






\begin{figure}[!htb]
 \begin{minipage}{0.5\textwidth}
\scriptsize 
\begin{tabular}{@{}p{1\textwidth}} 
 \hline 
  \multicolumn{1}{c}{(A) TextEditor} \\ \hline
  \vspace{-4mm}
\begin{Verbatim}[commandchars=\\\{\}, tabsize=2]
public class TextEditor extends JTextPane \{
 public UndoAction undoAction = new UndoAction();
 public RedoAction redoAction = new RedoAction();
 public CompoundUndoManager undo;
 public TextEditor(Workspace workspace) \{
  this.workspace = workspace;
  undo = new CompoundUndoManager(workspace);
  Action[] actionsArray = getActions();
  for (int i = 0; i < actionsArray.length; i++) \{
   Action a = actionsArray[i];
   actions.put(a.getValue(Action.NAME), a);
   \}
   actions.put("undo", undoAction);
   actions.put("redo", redoAction);
   \}
   public void discardUndoRedo() \{
    undo.discardAllEdits();
    undoAction.updateUndoState();
    redoAction.updateRedoState();
   \}
  public class UndoAction extends AbstractAction \{
   public UndoAction() \{
    super("Undo");
   \}
   public void actionPerformed(ActionEvent e) \{
    try \{
     undo.undo();
     updateUndoState();
     redoAction.updateRedoState();
    \} catch (CannotUndoException ex) \{
    \}\}
   public void updateUndoState() \{
    setEnabled(undo.canUndo());
   \}\}
   public class RedoAction extends AbstractAction \{
    public RedoAction() \{
     super("Redo");
    \}
    public void actionPerformed(ActionEvent e) \{
     try \{
      undo.redo();
      updateRedoState();
      undoAction.updateUndoState();
     \} catch (CannotRedoException ex) \{
     \}\}
      public void updateRedoState() \{
       setEnabled(undo.canRedo());
     \}\}
  \end{Verbatim}
      \vspace{-4mm}
     \\ \hline
\end{tabular} 
\caption{Result No 1: textmash:TextEditor}
\label{fig:textEditor}
\end{minipage}
\end{figure}





\begin{figure}[!htb]
 \begin{minipage}{0.5\textwidth}
\scriptsize 
\begin{tabular}{@{}p{1\textwidth}} 
 \hline 
  \multicolumn{1}{c}{(B) No. 5.6,8, ConsoleTextEditor} \\ \hline
  \vspace{-4mm}
\begin{Verbatim}[commandchars=\\\{\}, tabsize=2]
public class ConsoleTextEditor extends JScrollPane \{
  private UndoAction undoAction = new UndoAction();
    private RedoAction redoAction = new RedoAction();
    private TextUndoManager undoManager;
    public ConsoleTextEditor () \{
     this.undoManager = new TextUndoManager();
     doc.addUndoableEditListener(undoManager);
     undoManager.addPropertyChangeListener(undoAction);
     undoManager.addPropertyChangeListener(redoAction);
     doc.addDocumentListener(undoAction);
     doc.addDocumentListener(redoAction);
     InputMap im = textEditor.getInputMap(JComponent.IN_FOCUSED);
     KeyStroke ks = KeyStroke.getKeyStroke(InputEvent.CTRL, false);
     im.put(ks, StructuredSyntaxResources.UNDO);
     ActionMap am = textEditor.getActionMap();
     am.put(StructuredSyntaxResources.UNDO, undoAction);
  \}
 private class RedoAction extends UpdateCaretListener
  implements PropertyChangeListener \{
  public RedoAction() \{
   setEnabled(false);
  \}
  public void actionPerformed(ActionEvent ae) \{
   undoManager.redo();
   setEnabled(undoManager.canRedo());
   undoAction.setEnabled(undoManager.canUndo());
   super.actionPerformed(ae);
  \}
    public void propertyChange(PropertyChangeEvent pce) \{
     setEnabled(undoManager.canRedo());
        \} \}
 private class UndoAction extends UpdateCaretListener 
 implements PropertyChangeListener \{
   public UndoAction() \{
    setEnabled(false);
   \}
   public void actionPerformed(ActionEvent ae) \{
    undoManager.undo();
    setEnabled(undoManager.canUndo());
    redoAction.setEnabled(undoManager.canRedo());
    super.actionPerformed(ae);
 \}
  public void propertyChange(PropertyChangeEvent pce) \{
    setEnabled(undoManager.canUndo());
  \}\}
  \end{Verbatim}
      \vspace{-4mm}
     \\ \hline
\end{tabular} 
\caption{Result No 5,6,8: groovy: DocumentUndoManagerImpl}
\label{fig:textEditor}
\end{minipage}
\end{figure}




