


\begin{figure}[htb]
\footnotesize
      \begin{minipage}{0.5\textwidth}
      \begin{tabular}{@{}p{0.9\textwidth}}
   \\ \hline
  \multicolumn{1}{c}{(A) A partial program provided by users} \\ \hline
  \begin{Verbatim}[commandchars=\\\{\}, tabsize=2]
1.public OpenMapRealVector ebeDivide(RealVector v)\{
2. checkVectorDimensions(v.getDimension());
3. OpenMapRealVector res = new 
   OpenMapRealVector(this);
4. Iterator iter = res.entries.iterator();
5. while(/*Hole 0*/(Boolean)TedSketch.INVOKE(0)...)\{
6.  /*Hole 1*/ TedSketch.INVOKE(1)...; 
7. \}
8. return res;
9.\}
 \end{Verbatim}
        \\ \hline
  \multicolumn{1}{c}{(B) A JUnit test case that defines the correct functionality } \\ \hline
  \begin{Verbatim}[commandchars=\\\{\}, tabsize=2]
  @Test
1.public void testBasicFunctions() \{
2. RealVector  v_ebeDivide = v1.ebeDivide(v2);
3. double[] result_ebeDivide = \{0.25d, 0.4d, 0.5d\};
4. assertClose("compare vect", v_ebeDivide.getData(),
   result_ebeDivide, normTolerance);
5.\}
 \end{Verbatim}
   \\ \hline
  \multicolumn{1}{c}{(C) A solution generated by \tool that pass all test cases } \\ \hline
  \begin{Verbatim}[commandchars=\\\{\}, tabsize=2]
1. TedSketch.INVOKE(0): 
2.  iter.hasNext()
3. TedSketch.INVOKE(1): 
4.  iter.advance();
5.  Object o1 = iter.key();
6.  Object o2 = divideVal(v, itr);
7.  res.setEntry(o1, o2);
 \end{Verbatim}  
 \\  \hline
 \end{tabular}
  \end{minipage}
   \caption{An API synthesis example of non-straight-line code fragment  }
 \label{fig:motivate}

 \end{figure}