


\begin{figure}[t!]
\footnotesize
      \begin{minipage}{0.5\textwidth}
      \begin{tabular}{@{}p{0.9\textwidth}}
        \\ \hline
  \multicolumn{1}{c}{(B) A program sketch written by users under synthesis} \\ \hline
  \begin{Verbatim}[commandchars=\\\{\}, tabsize=2]
1. public void visit (...) \{...
2.  T type = ...;
3.  if ((Boolean) TedSketch.INVOKE(0)
     .addArgument(Object.class, type)...) \{
4.   TedSketch.INVOKE(1)
      .addArgument(Object.class, type)
      .setReturnType(null)...;
5.  \} else \{
6.  TedSketch.INVOKE(2)
     .addArgument(Object.class, type)
     .setReturnType(null)...;
7. \}
 \end{Verbatim}
   \\ \hline
  \multicolumn{1}{c}{(B) A sample test case that defines the program's correct behavior} \\ \hline
  \begin{Verbatim}[commandchars=\\\{\}, tabsize=2]
1. public void testTypedExterns() \{
2.  testSets(false, externs, js, output, 
     "\{alert=[[Foo.prototype]]\}");
3. \} 
\end{Verbatim}  
 \\  \hline
   \multicolumn{1}{c}{(C) A solution found by \tool that satisfies all test cases } \\ \hline
  \begin{Verbatim}[commandchars=\\\{\}, tabsize=2]
1. TedSketch.INVOKE(0): 
2.  typeSystem.isInvalidatingType(type)
3. TedSketch.INVOKE(1): 
4.  prop.invalidate();
5. TedSketch.INVOKE(2): 
6.  prop.addTypeToSkip(type);
\end{Verbatim}  
 \\  \hline
 \end{tabular}
  \end{minipage}
   \caption{ A synthesis task that uses arguments with generic types }
 \label{fig:generic}

 \end{figure}