\documentclass[10pt,conference]{IEEEtran}

%\usepackage{amssymb}
%\setcounter{tocdepth}{3}
%\usepackage{graphicx}

\usepackage{url} 
%\newcommand{\keywords}[1]{\par\addvspace\baselineskip
%\noindent\keywordname\enspace\ignorespaces#1}

%\usepackage{fancybox}
\usepackage{cite}
\usepackage{color,soul}
\usepackage{hyperref}
\usepackage{algorithmic}
\usepackage{flushend}
\usepackage{fancyvrb}
\usepackage[normalem]{ulem}
\usepackage{graphicx}
%\usepackage[lined, algonl, ruled, boxed, linesnumbered]{algorithm2e} 
\usepackage[linesnumbered,ruled,vlined]{algorithm2e}
\usepackage{amsmath}
\usepackage{amssymb} 
\usepackage{graphicx}
\usepackage{verbatim}
\usepackage{booktabs}
\usepackage{multirow}
\usepackage{listings}
\usepackage{subfigure}
\usepackage{latexsym}
\usepackage{xspace}
\usepackage[usenames,dvipsnames]{xcolor}
\usepackage{hyperref}
\usepackage{relsize}
\usepackage{chngpage}

\newcommand{\Comment}[1]{}
%\newcommand{\Section}[1]{\vspace*{-0ex}\section{#1}\vspace*{-0ex}}
%\newcommand{\Subsection}[1]{\vspace*{-0ex}\subsection{#1}\vspace*{-0ex}}

\newcommand{\tool}{\textsc{TedSketch}\xspace}
\newcommand{\sj}{\textsc{EdSketch}\xspace}
\newcommand{\spt}{\textsc{SyPet}\xspace}
\newcommand{\fixme} [1] {\textcolor{red}{{\it FIXME}: #1}}
\newcommand{\delete} [1] {\textcolor{red}{#1}}
\newcommand{\add} [1] {\textcolor{blue}{#1}}
% \newcommand{\javac}{javac\xspace} % This package lets you punctuate \javac normally and get good spacing, e.g., \javac.  gives you: javac.
\newcommand{\codefont}[1]{{\small{\texttt{#1}}}}
\newcommand{\todo} [1]{\textcolor{blue}{{\sf TODO}: #1}}
%\newcommand{\hlc}[2][yellow]{ {\sethlcolor{#1} \hl{#2}} }
%\newcommand{\link}[2]{\hyperref[#1]{\color{blue}\setulcolor{blue}\ul{#2}}
%\newcommand{\hlc}[2][yellow]{ {\sethlcolor{#1} \hl{#2}} }


\definecolor{mygreen}{rgb}{0,0.6,0}
\definecolor{mygray}{rgb}{0.5,0.5,0.5}
\definecolor{mymauve}{rgb}{0.58,0,0.82}

\definecolor{javared}{rgb}{0.6,0,0} % for strings
\definecolor{javagreen}{rgb}{0.25,0.5,0.35} % comments
\definecolor{javapurple}{rgb}{0.5,0,0.35} % keywords
\definecolor{javadocblue}{rgb}{0.25,0.35,0.75} % javadoc


\lstset{language=Java,
basicstyle=\ttfamily,
keywordstyle=\color{javapurple}\bfseries,
stringstyle=\color{javared},
commentstyle=\color{javagreen},
morecomment=[s][\color{javadocblue}]{/**}{*/},
numbers=left,
numberstyle=\tiny\color{black},
stepnumber=1,
numbersep=10pt,
tabsize=4,
showspaces=false,
showstringspaces=false}


%\hyphenation{op-tical net-works semi-conduc-tor}



\begin{document}

\title{Test-Execution-Driven Sketching for Complex APIs}


% author names and affiliations
% use a multiple column layout for up to three different
% affiliations
%\author{\IEEEauthorblockN{,  Sarfraz Khurshid}
%\IEEEauthorblockA{The University of Texas at Austin\\
%Email: \{, khurshid\}@utexas.edu}}

% make the title area
\maketitle

\begin{abstract}

  We introduce the \tool framework for synthesizing code fragments
  with respect to test cases that specify expected program behavior.
  Given a skeletal implementation, aka a \emph{sketch}, that has
  \emph{holes}, i.e., placeholders for desired fragments, and a test
  suite as inputs, \tool outputs sequences of Java APIs that
  complete the holes such that all tests pass.  \tool is based on a
  \emph{test-execution-driven} approach that effectively explores
  potentially very large spaces of candidate programs.  \tool executes
  each given test and uses the test executions to guide the
  exploration and prune large parts of the space.  Our focus is
  synthesis of method call sequences over given libraries.  Our
  approach handles sketches with non-straight-line code, e.g.,
  while-loops and if-else statements.  We embody our approach in a prototype tool, and perform an
  experimental evaluation using a number of sketching tasks for
  complex APIs.  We also compare \tool with \spt, which is a recently
  developed state-of-the-art tool using Petri nets and SAT
  solving, but is designed to only synthesize straight-line code.  The
  experimental results show that \tool can efficiently complete
  non-trivial sketches with non-straight-line code, and can sometimes
  even outperform \spt for subjects with straight-line code.

  \Comment{
To achieve a task of interest, developers always need to combine complex APIs together. Most tools that help programmers find code fragments follow a generate-and-validate style that first generate search space based on pre-defined or mined templates and validate these candidates against  correctness properties. 

To help programmers use APIs more easily, we present \tool,  a novel approach that dynamically generates program candidates for API sequences during the test execution and backtracks whenever the candidate encounters any test failures or runtime exceptions. We introduce effective pruning strategies to explore the actual program behavior with an execution-driven synthesis engine. Taken a partial program and a given test suite as the specification,  \tool completes the partial program with API sequences that satisfy all test cases. We further illustrate that based on execution-driven sketching, \tool could synthesize non-straight-line APIs with if-branches and while-loops.   Experimental results show that \tool compares well with another SAT-based API completion tool, SyPet, and moreover,  \tool could synthesize API sequences with if-branches and while-loops efficiently. 
}



\end{abstract}

% no keywords
\section{Introduction}


Reuse of existing code from class libraries is often difficult because it involves using unfamiliar APIs, and the  numerous combination of APIs makes the synthesis of API sequences even harder. There has been much research on helping developers explore APIs by mining existing code~\cite{strathcona:icse05, portfolio:icse13, jungloid:pldi05} based on probabilistic models~\cite{slang:pldi14, codehint:icse14}.  Some of them take  natural-language queries~\cite{sniff:fase09, freeQuery:oopsla15}   and synthesize a suitable code fragment that satisfies the query.   A recent work~\cite{isil:sypet17} uses SAT solver for graph-reachability analysis and synthesizes APIs based on the given test suite. Although the effectiveness of these techniques have been impressed, these techniques follow a similar generate-and-validate approach that generates all program candidates and validate them based on the correctness property, such as  partial specification~\cite{codehint:icse14} or test cases~\cite{isil:sypet17, testDriven:pldi14}. The key limitation for these two-phase synthesis techniques is that, without runtime information, the generation phase may construct a number of candidates  that are not feasible for a specific test execution. For instance, they may generate multiple APIs whose receiver objects are evaluated to be \codefont{null} or unreachable code fragments within an if-condition while the condition expression is evaluated to be \codefont{false}. 


%%More limitation for G&V and more benefit needed here
With the insight that an integration of generation and validation could help develop an effective approach for API synthesis, we present \tool, a new synthesis approach for complex APIs based on test-execution-driven sketching. 
As input, \tool takes a sketch (partial program) with holes for an API sequence, and a test suite that characterizes the expected behavior of this partial program. \tool supports API holes in  Java syntax for both conditions and statements. It executes the test suite against the sketch and backtracks  when it encounters a runtime exception or a test failure. \tool terminates when all API combinations have been explored or a solution has been found that satisfies the given test suite.  

 When the test execution reaches a ``hole'', \tool dynamically constructs candidates based on the runtime information, selects a candidate, continues executing this candidate, and validate this candidate using test execution. Yet if the test execution does not reach a ``hole'', e.g., the ``hole'' inside a while-loop while the while-condition is evaluated to be \codefont{false}, \tool will not create any new search space. \tool leverages a recent developed execution-driven sketching engine  called \sj~\cite{sketch4j:spin17}  as backend to handle backtracking search via re-execution~\cite{verisoft:97}.  

To initialize the search for API synthesis, \tool generates API candidates for the ``holes'' based on input types using reflection. These candidates are put in a candidate vector and \tool assigns a unique identifier for each candidate using its index in the candidate vector.  When \tool performs synthesis, it dynamically selects a candidate identifier for each ``hole'' using non-deterministic \codefont{choose()}  operator and executes the program based on the candidate it selects. 


The key novelty of our work is to introduce effective prioritization strategies to divide the search space into multiple sub-spaces with more constraints. The constraints are defined based on Java semantics and some heuristics for API synthesis. For instance, based on Java semantics,  if there does not exist any arguments or intermediate values with the same type of a parameter in the method, this method invocation is ignored as one of its parameters cannot be initialized. Moreover,  we introduce the notion of locality that arguments should be consumed as quickly as possible and the objects generated later in the sequence could use more information generated previously and thus are more likely to be used as return value. Leveraging the test-execution-driven sketching, \tool supports API synthesis for non-straight-line code fragments that have if-branches and while-loops. 

%Given that prior works complete API sequence based on the requirement that synthesized methods should use all of its inputs and any intermediate values should be produced~\cite{jungloid:pldi05, codehint:icse14, isil:sypet17}, \tool introduces several simple pruning strategies to optimize the API completion.  For a sequence of APIs, \tool prunes semantic-equivalent API sequences based on simple program isomorphic analysis and  prunes API calls with  non-consumed intermediate values based on the pre-defined assumption. 

%When \tool reaches a hole as the condition expression, it dynamically selects a method call that returns a \codefont{boolean} value, continues executing the program based on this selection, and backtracks if it encounters a runtime exception or test failures. 
 
To evaluate \tool, we first performed experiments to investigate its efficacy of synthesizing straight-line APIs compared to \spt, a state-of-the-art API synthesis tool based on SAT solver and advanced reachability analysis. We then investigate if our prioritization strategies could effectively expedite  the exploration of correct solutions. Lastly, we demonstrate how \tool handle APIs with if-else-branches and while-loops.  Existing works  do not directly support API synthesis with branches. We extend \spt to handle non-straight-line code synthesis and discuss the lessons we learn with this extension.

 The experimental results show that \tool's synthesis efficacy compares well with \spt. In particular, out of 47 synthesis tasks, \tool completes 39 subjects in an average of 52 seconds, which is very close to the synthesis efficacy of \spt, 42 subjects in an average of 49 seconds.  Our experiments also  illustrate that our prioritization strategies effectively reduce the performance time in 61\% of subjects. Moreover, \tool completes 10 sketching tasks with if-condition and while-loops from open source projects in a reasonable amount of time. 70\% of these non-straight-line synthesis tasks are completed in one minute with 5 methods sketched on average.  


 This paper makes the following contributions:
 
 \begin{itemize}
 \item \textbf{Test-execution-driven synthesis of API calls.}  We
   present a novel approach for synthesizing API calls based on
   test-execution-driven sketching. Given a partial program and a few
   test cases, \tool completes the partial program with API calls to
   conform to the expected behavior represented by the given test
   cases.
 \item \textbf{Pruning Strategies.  }  To enhance the efficacy of our
   approach, we introduce pruning strategies that reduce the search
   space of candidates based on the semantics of Java.  The
   experimental results show that our pruning strategies can
   significantly reduce the search space of API call synthesis.
 \item \textbf{API call synthesis with conditional branching and
   loops.  }  Our approach supports sketches that have conditional
   branches and while loops, and completes holes for conditions and
   statements using sequences of API calls.  \Comment{-driven
     sketching, \tool can successfully complete API call ``holes'' for
     conditions and statements.  To compare non-straight-line API
     synthesis with existing API completion tools, we extend \spt with
     test partition and demonstrate that \tool is more effective in
     synthesizing non-straight-line APIs than the extension of \spt.}
 \item \textbf{Experimental Evaluation.  }  We experimentally evaluate
   \tool and compare it with \spt, a recently developed
   state-of-the-art technique designed for synthesis of straight-line
   code. The results show that our approach can effectively handle
   non-trivial sketches with non-strainght-line code, and compares
   favorably with \spt for straight-line code.
  \end{itemize}


\Comment{
The rest of the paper is organized as followed: Section~\ref{sec:motivate} presents a motivating example to illustrate our  approach. Section~\ref{sec:approach} describes how we perform execution-driven sketching for API calls and how we prune the search space for efficient search, followed by the evaluation in Section~\ref{sec:eval}. We discuss related work in Section~\ref{sec:related} and conclude at Section~\ref{sec:conclude}.
}







\section{Motivating Examples}\label{sec:motivate}



\begin{figure}[htb]
\footnotesize
      \begin{minipage}{0.5\textwidth}
      \begin{tabular}{@{}p{0.9\textwidth}}
 \\ \hline
  \multicolumn{1}{c}{(A) A method that contains an API sequence under synthesis } \\ \hline
  \begin{Verbatim}[commandchars=\\\{\}, tabsize=2]
//Compute the ith  eigenvalue of a matrix
1.Vector2D eigenvalue(RealMatrix arg0, int arg1) \{
2. //Synthesize a sequence of APIs 
   //for desired functionality
3.\}
 \end{Verbatim}  
  \\ \hline
  \multicolumn{1}{c}{(B) A given test case that specifies desired behavior  } \\ \hline
  \begin{Verbatim}[commandchars=\\\{\}, tabsize=2]
1.public static boolean test0() throws Throwable \{
2. double[][] mat = new double[][]\{\{0,-20\},\{10,10\}\};
3. RealMatrix matrix = new Array2DRowRealMatrix(mat);
4. Vector2D result = eigenvalue(matrix, 0);
5. Vector2D target = new Vector2D(5, 5*Math.sqrt(7));
6. return Math.abs(result.getX()-target.getX())<1e-6 
 \&\& Math.abs(result.getY()-target.getY())<1e-6;
7.\}
 \end{Verbatim}  
  \\ \hline
  \multicolumn{1}{c}{(C) Input Sketch for \tool  Written by Users} \\ \hline
  \begin{Verbatim}[commandchars=\\\{\}, tabsize=2]
1.Vector2D eigenvalue(RealMatrix arg0, int arg1) \{
2. TedSketch.INVOKE(0) /* Hole ID*/
3. .addParameter(RealMatrix.class, arg0)
4. .addParameter(int.class, arg1)
5. .setReturnType(Vector2D.class)
6. .addPackage("org.apache.commons.math3.linear")
7. .invoke();
8.\}
 \end{Verbatim}
        \\ \hline
  \multicolumn{1}{c}{(D) Input Configuration JSON File for \spt Written by Users} \\ \hline
  \begin{Verbatim}[commandchars=\\\{\}, tabsize=2]
1. \{ 
2.  "methodName": "eigenvalue",
3.  "srcTypes": [	"RealMatrix", "int"],
4.  "paramNames": ["arg0", "arg1"],
5.  "tgtType": "Vector2D",
6.  "packages": ["org.apache.commons.math3.linear"],
7.  "testPath": "TestSource.java"
8.\}
 \end{Verbatim}
   \\ \hline
  \multicolumn{1}{c}{(E) A solution generated by \tool based on the test suite  } \\ \hline
  \begin{Verbatim}[commandchars=\\\{\}, tabsize=2]
1.TedSketch.INVOKE(0)
2. EigenDecomposition res0 = new 
    EigenDecomposition(arg0);
3. double res1 = res0.getImagEigenvalue(arg1);
4. double res2 = res0.getRealEigenvalue(arg1);
5. Vector2D res3 = new Vector2D(res2, res1);
6.return res3;
 \end{Verbatim}  
 \\  \hline
 \end{tabular}
  \end{minipage}
   \caption{An synthesis example of straight-line API sequence}
 \label{fig:background}
 \end{figure}



\begin{figure}[htb]
\footnotesize
      \begin{minipage}{0.5\textwidth}
      \begin{tabular}{@{}p{0.9\textwidth}}
   \\ \hline
  \multicolumn{1}{c}{(A) A partial program provided by users} \\ \hline
  \begin{Verbatim}[commandchars=\\\{\}, tabsize=2]
1.public OpenMapRealVector ebeDivide(RealVector v)\{
2. checkVectorDimensions(v.getDimension());
3. OpenMapRealVector res = new 
   OpenMapRealVector(this);
4. Iterator iter = res.entries.iterator();
5. while(/*Hole 0*/(Boolean)TedSketch.INVOKE(0)...)\{
6.  /*Hole 1*/ TedSketch.INVOKE(1)...; 
7. \}
8. return res;
9.\}
 \end{Verbatim}
        \\ \hline
  \multicolumn{1}{c}{(B) A JUnit test case that defines the correct functionality } \\ \hline
  \begin{Verbatim}[commandchars=\\\{\}, tabsize=2]
  @Test
1.public void testBasicFunctions() \{
2. RealVector  v_ebeDivide = v1.ebeDivide(v2);
3. double[] result_ebeDivide = \{0.25d, 0.4d, 0.5d\};
4. assertClose("compare vect", v_ebeDivide.getData(),
   result_ebeDivide, normTolerance);
5.\}
 \end{Verbatim}
   \\ \hline
  \multicolumn{1}{c}{(C) A solution generated by \tool that pass all test cases } \\ \hline
  \begin{Verbatim}[commandchars=\\\{\}, tabsize=2]
1. TedSketch.INVOKE(0): 
2.  iter.hasNext()
3. TedSketch.INVOKE(1): 
4.  iter.advance();
5.  Object o1 = iter.key();
6.  Object o2 = divideVal(v, itr);
7.  res.setEntry(o1, o2);
 \end{Verbatim}  
 \\  \hline
 \end{tabular}
  \end{minipage}
   \caption{An API synthesis example of non-straight-line code fragment  }
 \label{fig:motivate}

 \end{figure}
In this section, we define the problem of API synthesis and illustrate our approach using two examples of straight-line and non-straight-line code. Both examples use APIs from the \codefont{Apache-Math} library~\cite{math}. 

\subsection{Problem Definition}

API synthesis is an approach that generates a sequence of method invocations to perform a desired functionality based on the given input types, arguments, and the output type of the API sequence. For \tool, the desired behavior is defined as the satisfaction of all test cases. That is,  the correctness of the sketched solution from \tool is with respect to the given test suite. We present a usage scenario of API synthesis in Figure~\ref{fig:background} (A) that tries to create an API sequence to compute the $i^{th}$  eigenvalue of a matrix. Several test cases are provided as correctness criteria and we highlight one of them in Figure~\ref{fig:background} (B).  

To use \tool for API synthesis, users provide input arguments used in the method sequence, source types or libraries that specify the scope of the API exploration, and the output type of the API sequence, shown as Figure~\ref{fig:background} (C). 
To further illustrate the input arguments of existing API synthesis techniques, we show an input configuration file  for a state-of-the-art API synthesizer called \spt. Shown as Figure~\ref{fig:background} (D), \spt asks for similar information to identify correct sequences of method calls. Note that \spt is known as the state-of-the-art method synthesizer that compares favorably with other synthesis tools~\cite{codehint:icse14, insynth:cav11}. 

API synthesis tools return a desired method sequence with respect to the test suite.  Figure~\ref{fig:background} (D) presents a synthesized method sequence from \tool that pass all test cases. This solution is semantically identical to the result from \spt using the same test suite. 

\subsection{Non-Straight-Line Synthesis Example }

We further illustrate \tool's ability to  sketch API calls in non-straight-line code such as conditions and loops.  

%Figure~\ref{fig:motivate} (A) presents a code 
%\codefont{ebeDivide} method derived from the \codefont{OpenMapRealVector} class in the Apache-Math project. This \codefont{ebeDivide} method performs an element-by-element division for each element in the vector. Assume that users want to implement this method, and they want to implement this method using  three local variables \codefont{v}, \codefont{res}, and \codefont{iter}, and a \codefont{while} loop to traverse the vector. The users leave the condition of the \codefont{while} loop and its body to be synthesized by \tool. 

Figure~\ref{fig:motivate} (A)  shows a code skeleton (sketch) written by users. In this sketch, users specify the \codefont{while} condition (line 5) and the body of the \codefont{while} loop (line 6)  as ``holes'' that need to be synthesized. %Similar to other API completion tools~\cite{codehint:icse14, isil:sypet17}, users provide the input parameters and output type of the expected API sequence, and let API synthesis tool complete the method calls. 
Different from other API synthesis tools  that could only synthesize a single straight-line method sequence, \tool enables users to specify multiple holes in non-straight-line programs with if-conditions and while-loops. Using a suite of JUnit test cases, \tool tries to complete these holes via test execution. A JUnit test case is highlighted in Figure~\ref{fig:motivate} (B).  

%Using 3 visible variables in this example, there are 59 visible methods that can be chosen for the a single API invocation.  Even if we only consider candidates with up to 4 statements for the \codefont{while} loop body and 1 statement for the \codefont{while} loop condition, the space of possible complete programs can be as large as 12.1 millions. 

%We use the existing test suite from the \codefont{Apache-Math} project as the correctness criteria for the partial program, and set up a time out as 30 seconds to eliminate infinite loop. \codefont{Apache-Math} project  contains 3.6k JUnit test cases yet only one test case reaches the holes of this sketch.  For more efficient synthesis, we pick this test case to synthesize the partial program and validate the solution against the entire test suite. 

\tool dynamically selects method candidates for the sketch when it executes given test cases.  When \tool first reaches the \codefont{while} condition hole (line 5), it generates all API candidates based on the given input types using reflection and non-deterministically selects a candidate  with respect to the given output type \codefont{boolean}. If the selected candidate returns \codefont{false} for the \codefont{while} loop condition, the ``hole'' inside the while loop will not be executed thus no new search space will be initialized. When \tool first reaches the ``hole'' inside the while-loop at line 6, it  selects a method invocation in a non-deterministically manner and incrementally adds more APIs to the sequence until it finds the first solution that satisfies all test cases. Users may specify the number of statements in the synthesized sequence; by default, we set the bound of the API calls as 4 and make it configurable.  \tool backtracks its search if the current method sequence encounters a runtime error or a test failure, and selects the next candidate for a new execution.  

A naive search may explore a huge amount of program candidates. Considering hundreds of methods and multiple options for receiver objects and parameters of these methods, the search space for this problem can be as large as 9.4 billion.  \tool introduces effective prioritization strategies to divide the search space of API generation to multiple smaller spaces based on some heuristics such as the values generated later in the sequence are more likely to be used as the return object. 

%prunes a number of such candidates based on the output type and a similar assumption with other API completion tools that all intermediate values should be consumed. \tool also prioritizes candidates based on a set of heuristics such as variable locality, etc.

Figure~\ref{fig:motivate} (C) presents a solution generated by \tool with one method invocation for the \codefont{while} condition and 4 statements for the  body of the while-loop. This  solution is semantically identical to the original implementation in the library based on manual inspection, In this example, \tool finds the corrrect solution in 50 seconds (including compilation and test execution time) after exploring 7.8k candidates. 


  




\section{Approach}

To identify methods that implement concerns, I build a prototype which invokes Code Search Engine (CSE) API and analyzes the results from CSE using partial program analysis~\cite{partialProgram:OOPSLA08}. I select SearchCode~\cite{SearchCode} because it is an open source code search engine with over 7000 projects from  Github, Bitbucket, Google Code, and Sourceforge, with complete API documentations. To identify queried features in the returned source code,  I use the mean of TF-IDF weight  for each query term as a weighting factor and select top k (k=5) methods that are related to the given query.  This approach is similar to prior works that use  IR~\cite{Denys:FCA12} and NL analysis~\cite{Hill:FindConcept07} for feature location. I choose IR approach because other approaches require history or structural analysis that might not be feasible for partial program. TF-IDF =  $avg( \log (1 + f_{t,d}) \times  \log \frac {N} {n_t}), f_{t,d}$ is the frequency of term $t$ in method $d$, $N$ is the total number of methods, $n_t$ is the number of methods that have the term $t$.

\section{Evaluation}\label{sec:eval} 

\begin{table*}[!htb]
\footnotesize
\begin{center}
\caption{ Straight-Line Subjects}
\label{table:line}
\begin{tabular}{@{}c|l|r|rr|rr|rr|rrrrrrrrrrrr@{}}

\multirow{3}{*}{ID}&	\multirow{3}{*}{Description}&\multirow{3}{*}{\#Tests}&\multicolumn{2}{c|}{\spt}&\multicolumn{6}{c}{\tool} \\
&&&\multirow{2}{*}{\#API}& \multirow{2}{*}{Space} &\multirow{2}{*}{\#API}&\multirow{2}{*}{Space}&\multicolumn{2}{c|}{With Ranking}&\multicolumn{2}{c}{Without Ranking}\\ 
&&&&&& &Time(s)&\#Run&Time(s)&\#Run\\ \midrule
1&Compute the pseudo-inverse of a matrix&1&3&41&3&1.7b&6&1.8k&12&5.3k\\
2&Compute the inner product between two vectors&1&3&24&2&258.6m&2&124&3&134\\
3&Determine the roots of a polynomial equation&1&3&35&3&23.1t&109&9.0k&380&69.3k\\
4&Compute the singular value decomposition of a matrix&1&3&24&3&45.5b&3&286&8&993\\
5&Invert a square matrix&1&3&28&2&12.7m&2&135&3&217\\
6&Solve a system of linear equations&1&6&115&-& $10^{22}$&-&146.7k&-&959.7k\\
7&Compute the outer product between two vectors&1&4&37&2&258.6m&2&117&3&127\\
8&Predict a value from a sample by linear regression&2&3&244&3&18.1b&517&3.7k&4&3.8k\\
9&Compute the ith eigenvalue of a matrix&2&-&$\perp$&4&$10^{15}$&62&211.2k&339&850.3k\\
10&Scale a rectangle by a given ratio&1&4&36&3&502.0b&314&32.0m&710&18.3m\\
11&Shear a rectangle and get its bounds&1&4&36&3&502.0b&308&31.9m&133&13.1m\\
12&Rotate a rectangle about the origin by quadrants&1&4&17&3&32.2b&2&907&3&2.6k\\
13&Rotate 2-D shape by the specified angle about a point&2&4&33&-&$10^{15}$&-&19.7m&-&18.2m\\
14&Perform a translation on a rectangle&1&4&26&3&502.0b&308&32.2m&-&55.0m\\
15&Intersect a rectangle and an ellipse&1&3&11&2&10.1m&2&40&1&3\\
16&Compute number of days since a date&2&3&22&3&45.8b&37&80.8k&70&111.6k\\
17&Subtract two dates considering timezone&3&4&408&3&2.3t&56&150.4k&122&214.2k\\
18&Determine if a year is a leap year&3&4&68&3&45.8b&24&102.3k&3&1.2k\\
19&Return the day of a date string&2&3&11&3&2.5t&42&252.9k&43&57.7k\\
20&Find the number of days of a month in a date string&2&4&83&-& $10^{16}$&-&38.1m&-&5.6m\\
21&Find the day of the week of a date string&2&4&44&-& $10^{16}$&-&19.4m&-&5.4m\\
22&Compute age given date of birth&2&3&30&3&45.8b&63&158.6k&86&174.1k\\
23&Compute the offset for a specified line in a document&1&3&19&3&1.7t&5&1.2k&11&4.0k\\
24&Get a paragraph element given its offset in the a document&1&3&24&3&1.7t&4&1k&11&3.8k\\
25&Obtain the title of a webpage specified by a URL&1&3&110&3&1.1b&35&10.7k&35&19.2k\\
26&Return doctype of XML document generated by string&1&6&21&-& $10^{19}$&-&5.4m&-&5.6m\\
27&Generate an XML element from a string&1&6&24&-& $10^{19}$&-&3.8m&-&4.8m\\
28&Read XML document from a file&1&3&14&3&5.5b&4&3.9k&5&2.8k\\
29&Generate an XML from file and query it using XPath&1&6&78&-& $10^{26}$&-&3.2m&-&3.3m\\
30&Get the value of root attribute from a XML file&1&5&17&-&$10^{18}$&-&13.0m&-&13.8m\\ \midrule
31&Check if a point is inside a rectangle&8&5&66&1&3.1k&0.5&1&0.4&1\\
32&Check if a line segment intersects a rectangle.&8&-&$\perp$&2&63.1m&1&224&2&265\\
33&Compute number of minutes between two time&8&-&$\perp$&2&175.5m&1&183&1&350\\
34&Get number of seconds since the midnight of sometime&8&2&15&1&3.5k&1&2&0.7&2\\
35&Compute the transpose of a matrix&8&3&12&3&1.7b&1&88&1&401\\
36&Compute the sum of two matrices&8&4&14&4&911.1t&99&3.8k&1.4k&56.5k\\
37&Compute exclusive or between an area and a rectangle&2&-&-&2&10.1m&3&135&3&136\\
38&Create an element with given name&3&4&18&4&3.3t&5&6.5k&1&779\\ \midrule
39&Calculate absolute value of an integer&8&1&15&1&2.4k&0.5&5&0.1&10\\
40&Increment an integer and return its old value&8&4&5&3&1.9k&0.4&60&0.9&28\\
41&Increment an integer by 2 and return its old value&8&-&-&5&271.5k&7&115.9k&0.5&110\\
42&Get the class name of an object&8&2&30&2&5.9m&0.5&25&0.6&28\\
43&Get the first value of an integer array&8&2&4&2&210&1&14&0.8&8\\
44&Calculate minimum value between two integers&8&1&14&1&7.1k&0.5&49&0.5&49\\
45&Calculate minimum value between three integers&8&2&25&2&179.8m&1.1&3.8k&1&4.3k\\
46&Given an array set the last entry the value of first entry&8&4&154&5&579.2k&1&3.1k&2&16.4k\\
47&Sort an integer array&8&1&14&1&7.6k&0.8&1&1&1\\


  \end{tabular}
  
  \vspace{1mm}
 $\perp$ represents out of memory. - represents time out after 30 minutes. The first 30 subjects are from the evaluation benchmark of \spt, 31-38 are 8 subjects derived from the open source projects used in \spt evaluation, and the rest are small synthesis tasks such as absolute value calculation.
\end{center}
\end{table*}

We evaluate \tool with two benchmarks that consist of straight-line and non-straight-line subjects. We address the following research questions in the evaluation: 

\begin{itemize}
\item How well does \tool perform on synthesizing straight-line code snippets involving Java APIs compared to the state-of-the-art API synthesizer?
\item How do the  prioritization strategies affect the search space and performance of the synthesis?
\item How effective is \tool to synthesize program fragments from open source projects with if-conditions and while-loops?
\end{itemize}

All performance experiments are conducted on a MacBook Pro with 2.7 GHz Inter core i5 processor and 8GB memory running OS X version 10.12.4. %The maximum heap memory is set as 2 Gigabytes. 


\subsection{Synthesizing Straight-Line API sequences}
To evaluate \tool, we first curate a benchmark of 47 synthesis tasks for straight-line API sequences, and compare \tool's synthesis efficacy with \spt~\cite{isil:sypet17}, a SAT-based API synthesizer with reachability analysis,  using the same benchmark.  Within these 47 subjects, 30 of them are used in the original evaluation of \spt, 8 are derived from open source projects used in \spt evaluation, and the rest involves small synthesis tasks like array sorting. The 30 synthesis tasks used in \spt evaluation are collected from 4 open source projects based on {\it StackOverflow} online forum and Github repositories, and the corresponding test cases are manually created in an incremental manner until \spt can find a correct solution with respect to the original implementation. For the 17 newly-added subjects, we follow a similar approach to create test cases. As we cannot replicate the experiments using the same machine, we execute \spt under our experiment setting (with smaller memory compared to their machine) and report the comparison result. Therefore, the result might be different from that reported in their original paper. To ease the comparison, we set up a time limit as 30 minutes for our experiments, the same as the default setting of \spt. 







\noindent{\textbf{Evaluation Results. }}  Table~\ref{table:line} reports the 47 subjects with brief descriptions, the numbers of provided test cases for the synthesis (\textit{\#Tests}), and the number of synthesized  APIs in the first solution (\textit{\#API}).   Column {\it Space} shows the search space of API sequence candidates with respect to the identified solution.  If \tool collects a total of $m$ methods from input types using reflection, the average number of parameters for these methods is $ap$, and the number of given arguments is $v$,  we calculate the search space of API sequence candidates as $\sum_{1}^{N} (m \times v^{ap})^i$ while $N$ is the number of synthesized APIs  in the correct solution. We define the search space as is because each argument in a method candidate can have a maximum of $v$ options, and we incrementally insert more APIs into the sequence, searching for a desired solution.   Note that it is just an estimated search space because API candidates are dynamically generated based on previous APIs and constraints, and not all candidates in the search space will be executed.  Column   \textit{\#Run} shows the number of executed program candidates when \tool finds the first solution that passes all test cases.  Column \textit{Time} represents the total performance time including program compilation and test execution when \tool finds the first solution. 

\tool successfully synthesizes 39 subjects with an average of 52 seconds. We manually investigate the tasks that \tool could not generate a solution within one hour, and find that these outliers usually have a relatively large distance for the intermediate value consumption, thus \tool de-prioritizes these candidates and times out after the preset time limit. For instance, shown as Figure~\ref{fig:outlier}  (A), the input argument \codefont{arg1} is not consumed until the $6^{th}$ statements, indicating that the maximum distance of the value consumption is 6. Therefore, \tool fails to prioritize the correct solution and identify it within the time limit.  Regarding the performance of the API synthesis,   out of 34 subjects that both tools could generate desired code, \tool outperforms \spt in 25 subjects. %indicating that \tool performs generally faster than the start-of-the-art. 






\begin{figure}[t!]
\footnotesize
      \begin{minipage}{0.5\textwidth}
      \begin{tabular}{@{}p{0.9\textwidth}}
        \\ \hline
  \multicolumn{1}{c}{(B) A subject that \tool times out (No.6) } \\ \hline
  \begin{Verbatim}[commandchars=\\\{\}, tabsize=2]
1.public static double[] solveLinear
   (double[][] arg0, double[] arg1) \{
2. RealMatrix v1 = MatrixUtils.createMatrix(arg0);
3. RealMatrix v2 = v1.transpose();
4. LUDecomposition v3 = new LUDecomposition(v2);
5. DecompositionSolver v4 = v3.getSolver();
6. RealMatrix v5 = v4.getInverse();
7. double[] v6 = v5.preMultiply(arg1);
8. return v6;
9. \} 
\end{Verbatim}
   \\ \hline
  \multicolumn{1}{c}{(B) A subject that \spt times out (No.41)} \\ \hline
  \begin{Verbatim}[commandchars=\\\{\}, tabsize=2]
1.public int getAndAddTwo(IntegerWrapper arg0)\{
2. int res0 = arg0.get();
3. arg0.increment();
4. arg0.increment();
5. return res0;
6.\}
\end{Verbatim}  
 \\  \hline
 \end{tabular}
  \end{minipage}
   \caption{Two ``timeout'' subjects for \tool and \spt }
 \label{fig:outlier}

 \end{figure}


On the other hand,  \spt finds desired API sequences for 42 subjects with an average of 49 seconds. Figure~\ref{fig:outlier}  (B)  presents a subject that \spt fails to identify a correct solution within the time limit whereas \tool detects the solution in 7 seconds. We are not aware of the root cause of this failure as the source code of \spt is not publicly available.  In addition,  \spt throws out of memory exception in 3 synthesis tasks, indicating that constructing a large reachability graph using SAT solver can consume a large memory.   \tool does not require reachability analysis and the memory \tool uses for API synthesis is linear with respect to the number of method invocations under sketching. 


In summary, \tool compares generally well with the state-of-the-art API synthesizer to sketch straight-line API sequences. 


\subsection{Efficacy of Prioritization Strategies}\label{sec:strategyEval}

To evaluate if our prioritization strategies can expedite the search of the desired API sequence, we report the number performance time and the number of executed programs with and without prioritization strategies when \tool finds the first solution that satisfies all test cases.  The last 4 columns of Table~\ref{table:line} lists these 4 numbers for all 47 synthesis tasks in our benchmark. 

Within the 38 synthesis tasks that \tool can find desired solutions both with and without prioritization strategies, \tool with prioritization outperforms 23 tasks.    
  We further use the Spearman test to measure if the number of executed program candidates is significant different with and without prioritization strategies and the $p<0.01$ indicates that our prioritization strategies significantly reduce the number of executed candidates. We manually inspect the subjects in which prioritization strategies perform poorly and highlight one example in Figure~\ref{fig:outlier} (B).   In this example, the return object comes from the first statement and the desired sequence contains two repetitive APIs, which contradicts with two heuristics applied in the prioritization strategies. 

The experiment results also indicate that even without pruning strategies, our test-execution-driven sketching could identify a desired API sequence by executing only a very small portion of the total search space (less than $0.001\%$). 




\begin{table*}
\footnotesize
\begin{center}
\caption{ Non-Straight-Line Subjects}
\label{table:branch}
\begin{tabular}{@{}c|c|l|rrrrrrrr@{}}
Project&ID&Description&\#Cond 	&\#Stmt&Space&\#Tests&\#Run&Time(s)\\ \midrule
Chart&1&\codefont{createAndAddEntity()}:  Created an entity for the axis&3if&6&5.5$\times10^{21}$&1&279&9\\
90k; 2.2k&2&\codefont{addBaseTimelineException()}:  Adds a segment &1wh&1&2.4k&2&2&11\\ \midrule
Math&3&\codefont{ebeDivide()}:  Element-by-element division in a vector&1wh&4&9.4b&1&7.8k&50\\
Loc: 85k&4&\codefont{add()}: Optimized method to add two RealVectors&2if, 1wh&6&8.4$\times10^{19}$&1&170.5k&51\\
\#Tests: 3.6k&5&\codefont{setEntry()}: Set entry in specified row and column&3if&2&9.3$\times10^{14}$&1&529&33\\  \midrule
&6&\codefont{visit()}:   sets properties as ineligible&1if&2&9k&57&1&3\\
Closure&7&\codefont{applyCollapses()}:  Collapse  variable declarations &1if, 1wh&2&12.5b&6&125.0k&220\\
Loc: 90k&8&\codefont{remove()}: Remove this node&1if&3&1.4$\times10^{11}$&10&172.4k&333\\
\#Tests: 7.8k&9&\codefont{flattenReferences()}:  Flattens to collapsible properties&2if&2&7.0m&23&9&6\\
&10&\codefont{removeVar()}:  remove var if it has been coalesced&3if&3&1.8$\times10^{14}$&28&195.1k&2.3k\\
  \end{tabular}
  \vspace{1mm}
  
  
  *if represents the total number of synthesized APIs in if-conditions, and *wh represents the total number of generated methods in while loops.
\end{center}
\end{table*}



\subsection{Sketching Non-Straight-Line Code Fragments}~\label{sec:branch}

We further illustrate \tool's ability to synthesize APIs in if-conditions and while-loops that could hardly be  handled by existing API synthesizers.  
We select 10 non-straight-line code fragments from the implementation of 3 widely-used open source Java projects: \codefont{JFreeChart}~\cite{chart}, \codefont{Apache-Math}~\cite{math}, and \codefont{Closure} compiler~\cite{closure} for Javascript.  To distinguish from straight-line API synthesis, we select these subjects based on the criteria that  1) they should contain method invocations in both conditions and bodies of the conditions/loops; 2) at least one test case in the test suite should covered both the conditions and the bodies of the conditions/loops. The first rule ensure that \tool will synthesize APIs in the conditions,  and the second rule makes sure that there exist at least two holes in a single test execution and at least two holes will be sketched. 
Different from the straight-line benchmark whose test cases are hand-made by the authors of \spt or by us with the purpose of using libraries, the test suites for non-straight-line benchmark are directly from the open source projects with the purpose of validating these tools, which are usually written in JUnit test framework. %In particular, the test cases in the \codefont{Closure} project are organized in a non-conventional way using scripts rather than standard JUnit test cases. Our test-execution-driven synthesis approach does not have any requirement on the format of test cases and can be applied to any projects that can be executed. 


We manually create program sketches  and introduce ``holes'' for API sequences based on the original implementation. We execute all test cases that reach the ``holes'' to synthesize program sketches using API calls and   validate generated solutions against the entire test suite. %Finally, we manually inspect if the solutions are semantically equivalent to  the original programs. 
%Given that the subjects we select contain ``holes'' in both conditions and bodies of the conditions/loops, \tool will encounter at least two ``holes'' in  a single test execution, which is different from existing API completion tools.  We have tries to extend existing API synthesizers with test partition for synthesizing non-straight-line programs, and more details will be described in the Section~\ref{sec:discuss}. 


\noindent{\textbf{Evaluation Results. }}   Table~\ref{table:branch}  lists these 10 non-straight-line subjects, including basic information of the open source projects (lines of code {\emph Loc} and the number of test cases {\emph \#Tests}), a brief summary of each synthesis task (Column {\it Description}),  and the number of test cases that reaches the ``holes'' (Column  {\it \#Test}). Column {\it \#Cond} represents the number of invocations in conditions, while {\it 2if} indicates that \tool synthesizes a total of 2 method invocations in all if-condition expressions. It can be an API chain of two methods in an if-condition, 2 chained if-conditions, or nested if-conditions. Similarly, we use {\it 1wh} to represent  a while-condition with one API call. The column {\it \#Stmt} represents the total number of invocations synthesized in the body of the conditions/loops, yet these invocations may scattered in different branches.  %We have shown two  non-straight-line subjects in  Figure~\ref{fig:motivate} (No.3) and Figure~\ref{fig:generic}  (No.6). % {\it 2if, 1wh} represents that \tool synthesizes 2 method invocations in if-conditions and  one API call in while-condition. 



On average, \tool explores 67.2k program candidates before it finds a sequence of method calls that satisfies all test assertions  in a reasonable amount of time that 70\% of synthesis tasks can be done in 1 minute. We also see that \tool does not require many test cases to synthesize an API sequence. Particularly, some tasks with multiple holes in both while-condition and while-body can be synthesized using a single JUnit test case such as the subject No.3 shown in Figure~\ref{fig:motivate}. %Note that  we directly use the JUnit test cases from open source projects to synthesize the ``holes'' in open source projects, and we only report the number of API invocations that have been reached and synthesized by \tool.  \tool generates an implementation that performs the desired functionality in a reasonable amount of time as Y seconds on average to complete a synthesis task with an average search space of M million program candidates. 




In summary, \tool could successfully synthesize multiple API calls with branches and loops based on several test cases, while existing API synthesizers could only handle straight-line code. %We further investigate the efficacy of \tool to synthesize straight-line code in the next section.


\noindent{\bf Threats to Validity.}  We use  test cases as the
correctness criteria, which can generate plausible solutions that pass
all test cases but are not equivalent to some hypothetical correct
ones.  It can be incomplete to use test cases as correctness
specification.  When we evaluate non-straight-line subjects, we reuse
the original test cases provided by the open source projects, which
are not created with the purpose of code synthesis\Comment{. Although
  benefiting from the reuse of existing JUnit test cases, \tool may
  inadvertently generate plausible sketching results, i.e., passing
  all test cases but is incorrect from the perspective of users}. In
our experiments, we manually inspect the first generated solution for
each subject to validate its correctness. %We also list our evaluation dataset at~\cite{hua:eval} for  cross validation.

%\noindent{\textbf{External Validity.}}  
We only compare \tool with one state-of-the-art API synthesizer; thus,
our comparison result might not extend to other such tools, e.g.,
\textsc{CodeHint}~\cite{codehint:icse14} and
\textsc{InSynth}~\cite{insynth:cav11}.  We note that \spt has been
shown to be a state-of-the-art tool in comparison with both
\textsc{CodeHint} and \textsc{InSynth} using the same 30 subjects.  We
also include these 30 subjects in our comparison and \tool could
successfully synthesize most of them. 

 % \tool only uses given arguments as parameters and does not support features such as field dereferences. We envision that we could extend \tool to support field dereferences and enum types with reflection. % and insert them into the candidate vector for parameters. 


  \vspace{-2mm}
\section{Discussion}~\label{sec:discuss}
  \vspace{-2mm}
% Test Partion SyPet extension
%May not be correct solution
%\noindent{\bf Extending SyPet to handle non-straight-line code synthesis.} 
  
   The use of Petri nets in SyPet is effective, although
limited to synthesis of straight-line code.  It is possible in
principle to enhance SyPet to handle a broader class of synthesis
problems, e.g., those that \tool handles.  To illustrate, consider
using SyPet to synthesize parts of an ``if-else'' statement.  \Comment{

With the notion that the problem of API synthesis with if-else branches can be divided into some subproblem of straight-line API synthesis, we extend \spt, trying to support  non-straight-line API synthesis with test partition. 
}
Intuitively, the synthesis problem can be divided into three
subproblems of sketching the if-condition, the if-body, and the
else-body. Given that a test case can only exclusively execute either
the if-block or the else-block, we try to partition the given test suite
and synthesize these two blocks and the if-condition based on two
subsets of test cases. We enumerate all combinations of given test
cases, searching for an adequate test partition such that \spt could
generate API sequences for both the if-body and the else-body based on
these two subsets correspondingly. If the test partition can
successfully generate two method sequences for the if-body and the
else-body, we further collect test oracles for the if-condition with
respect to the tests, and let \spt synthesize an API sequence that
could generate this test partition using the if-condition. Based on
this idea of dividing the non-straight-line synthesis task to multiple
straight-line synthesis problems, we consider an extension of \spt to
synthesize chained if-conditions by splitting the test suite into a
few subsets.

To illustrate this extension based on test partitioning, consider the
method in Figure~\ref{fig:partition} that tries to classify the
triangle based on its three edges. The method \codefont{classify}
takes the lengths of three edges as input and returns the triangle's
classification as either acute, right angled, or obtuse.
Figure~\ref{fig:partition} (A) presents a method skeleton of
non-straight-line program with chained if-conditions. This method
skeleton is regarded as the test driver for the extension of \spt.  We
also manually provide more than 20 test cases for this synthesis task
to sketch 3 bodies and 2 if-conditions. Figure~\ref{fig:partition} (C)
shows a solution from the extension of \spt based on the test
partition.

We learn several lessons from extending an existing straight-line API synthesizer to support non-straight-line code. First,  it can be rather expensive to find an adequate test partition  that satisfies constraints from both condition bodies and if-conditions. Given 20 test cases, the possible partitions can be as large as $3^{20}$. Even though \spt successfully finds a desired API sequence for a subset of test cases, it might fail to sketch a condition with method invocations that could separate the test suite to these subsets. Second, this idea assumes that the synthesized APIs for the if-condition cannot change the current program state, i.e, they must be pure methods. Otherwise the idea of test-partition may fail due to the side effect of the if-condition. Lastly, the test-partition-based non-straight-line code synthesis can fail in synthesizing while-loops. 

While the simple idea of dividing the non-straight-line API synthesis
problem to several straight-line API synthesis sub-problems based on
test partitioning may not allow SyPet to effectively handle
non-straight-line synthesis problems, Petri nets could still provide
valuable guidance during search.  We believe future work on
integrating Petri nets with \tool holds promise in further optimizing
\tool.

\Comment{On the other hand, however, \tool does not require any test partition and can easily support  API synthesis with if-conditions and while-loops based on test-execution-driven sketching. }




\begin{figure}[t!]
\footnotesize
      \begin{minipage}{0.5\textwidth}
      \begin{tabular}{@{}p{0.9\textwidth}}
   
   \\ \hline
  \multicolumn{1}{c}{(A) A code skeleton under synthesis as the input of \spt} \\ \hline
  \begin{Verbatim}[commandchars=\\\{\}, tabsize=2]
1.String classify(int a, int b, int c) \{ 
2. if(condition1(a, b, c)) \{
3.  return body1();
4. \} else if(condition2(a, b, c)) \{ 
5.  return body2();
6. \} else \{
7.  return body3();
8. \} \}
 \end{Verbatim}
   \\ \hline
  \multicolumn{1}{c}{(B) Sample test cases that define correct behavior} \\ \hline
  \begin{Verbatim}[commandchars=\\\{\}, tabsize=2]
public static boolean test1() throws Throwable \{
 return classify(78, 79, 80).equals("acute");
\}
public static boolean test2() throws Throwable \{
 return classify(12, 13, 5).equals("right");
\}
public static boolean test3() throws Throwable \{
 return classify(10, 15, 10).equals("obtuse");    
\}
 \end{Verbatim}  
      \\ \hline
  \multicolumn{1}{c}{(B) A solution from extended \spt  } \\ \hline
  \begin{Verbatim}[commandchars=\\\{\}, tabsize=2]
1. body3(a,b,c): Library.obtuse(a,b,c);
2. body2(a,b,c): Library.right(a,b,c);
3. body1(a,b,c): Library.acute(a,b,c);
4. condition2(a,b,c): Library.isRight(a,b,c);
5. condition1(a,b,c): Library.isAcute(a,b,c);
 \end{Verbatim}
 \\  \hline
 \end{tabular}
  \end{minipage}
   \caption{Example of \spt extension to synthesize non-straight-line APIs   }
 \label{fig:partition}

 \end{figure}

  
  


%\noindent{\textbf{Internal Validity.}} 



%We believe that the reachability analysis used in \spt is complementary to ours and has the potential to further prune the search space of API sequences.  
%\noindent{\textbf{Construct Validity.}} 
%For our benchmark of straight-line subjects, the API completion tasks  are  defined 
%
% test cases are  not provided 
%
% \tool may generate a different solution that also satisfies all test cases compared to the synthesized API sequence generated by \spt. We regard solutions that pass all test cases are regarded as correct solution, we regard 
%In our performance comparison experiment, we manually transform the Java program to Sketch language, which may have inadvertently introduced behavioral differences. We list all subjects and tests at~\cite{hua:eval} for cross-validation.



\section{Related Work}

\paragraph{Corpus-wide Naming Rules}  Abebe et al.\/ investigate identifier quality  based on a catalog of Lexicon Bad Smells~\cite{abebe:wcre12}. 
Caprile and Tonella leverage pre-defined syntax and identifier dictionary to replace terms with their synonyms to ensure naming consistency~\cite{caprile:reconstruct}. Dei{\ss}enb{\"{o}}ck and Pizka proposed a formal model for concise and consistent naming, which requires a mapping from a concept domain to a lexical domain~\cite{ccnames:Pizka06}. Lawrie et al.\/ use hard-coded naming rules such as \textit{an identifier should not include another identifier in its name}~\cite{syntacticcc:SCAM06}.  Arnaoudova et al.\/ define a family of linguistic anti-patterns such as \textit{set* methods should not return an object} and investigate misunderstanding caused by anti-patterns~\cite{antipattern:CSMR13}. None of these approaches detect project-specific naming inconsistency.  
Allamanis et al. \/ build a tool called Naturalize to learn and enforce coding conventions~\cite{birdfse14:convension}. It uses n-gram based statistical natural language processing model to suggest natural identifier names and formatting conventions at the level of type names, method calls, and variable names. 
%  The only overlapping target in scope is the variable name recommendations reported by Naturalize. We run Naturalize on the same 39 projects using its default settings and compare variable name reports. We use a few examples to illustrate the difference between \niche and Naturalize. Naturalize misses inconsistent object names such as `\codefont{repStore}', which violates the commonly used name pattern (`\codefont{*FS}') found by type-use pattern analysis in \niche. On the other hand, Naturalizes suggests renaming `\codefont{local\_folder}' to `\codefont{localFolder}'. \niche does not produce such naming suggestion, as it focuses on checking name consistency based on the code's semantic functionality.
H{\o}st et al. \/ build a tool called Lancelot to find poor method names by comparing syntactic code features, such as `\textit{contains a loop}'~\cite{nameBug:ECOOP09} .  
%The only overlapping scope for Lancelot and \niche is the inconsistent method names found by \niche and the `\textit{method naming bugs}' from Lancelot. Lancelot misses inconsistent method names such as `\codefont{setSyncFlag}' which does not change the field value because it checks syntactic features only without considering the purity of a method. Lancelot reports  `\codefont{setExit}' method as a naming bug, since it returns the object itself (\codefont{return this}), but in this case, \niche classifies it as consistent because the method is impure. 
Buse and Weimer develop a readability metric for Java by training data from human annotators~\cite{buse:readability}. They find that their readability metric exhibits a significant level of correlation with static analysis warnings found by FindBug~\cite{FindBug:OOPSLA06}. Though their work connects the readability of identifier names to  static analysis warnings, they do not study the correlation between inconsistent names and bug fixes. 

\paragraph{Defect Study}  Abebe et al.\/ find correlation between naming rule violations and  coupling metrics~\cite{abebe:wcre12,ckmetrics:oopsla91}. Butler et al.\/ find correlation between FindBug's static analysis warnings and violations of hard-coded rules~\cite{FindBug:OOPSLA06,butler:csmr10}. Since static analysis tools suffer from 30\% to 100\% false positive rates~\cite{kremenekWarn:fse04}, we cannot conclude whether naming inconsistency is correlated with actual defects. Boogerd et al.\/ find that only 10 out of 88 hard-coded naming rules have positive correlation with defect density~\cite{boogerd:icsm08}. We differ from prior studies by mining rules from a corpus and studying how these rule violations relate to actual defects. Our study also finds that the confusion and misunderstanding caused by inconsistent name may propagate to their callers and that the lifetime of inconsistent names is shorter than the rest. 

%\paragraph{Mining Usage Patterns} 
 
\paragraph{Feature Location} Poshyvanyk et al. \/ use information retrieval approach to locate a queried feature in source code~\cite{Denys:FCA12}. They  evaluate the similarity between documents and user query and cluster the source code based on formal concept analysis.  Portforlio~\cite{Portfolio:DenysICSE11} and Export~\cite{Export:DenysASE13} identify related functions by combining both latent structure similarity and lexical information similarity. Our feature location approach is similar to~\cite{Portfolio:DenysICSE11} yet we focus on suggesting structured implementation for the feature rather than identify feature location.
Rastkar et al. \/ summarize the structure of multiple instances of a crosscutting concern in natural language, yet they only extract structural facts in the level of method signature and class hierarchy~\cite{Murphy:nlConcern11}. We make it one step further to suggest structured feature implementation based on  user query.

\paragraph{Code Search} Our example clustering approach is similar to some prior works that extract representative examples for specific APIs or user query. 
MAPO~\cite{MAPO:ECOOP09} leverages frequent call sequences to cluster the usage of specific APIs and rank abstract usage patterns based on the context similarity. Buse et al\/\cite{Buse:apiICSE12} propose to generate abstract API usages by synthesizing code examples using symbolic execution for a particular API. Different from these works that generate abstract usage patterns for specific API or data type, SNIFF~\cite{sniff:Sen09} performs type-based intersection of code chunks based on the keywords in the free-form query and cluster the common part of the code chunks for concrete code examples. However, these works   only focus on providing code examples based on the popularity or textural relevance while developers have to manually resolve structural dependencies before reusing the examples.  MUSE~\cite{MUSE:MarcusICSE15} addresses this limitation using slicing to generate concrete usage examples and selects the most representative ones based on the popularity and readability while  Keivanloo et al\/\cite{spotWork:ICSE14} uses clone detection to cluster examples involving loops and conditions.  There exists a number of code example suggestion tools that recommend call chains~\cite{Mandelin:jungloid05, parseWeb:ASE07, Xsnippet:OOPSLA06}  or contexts~\cite{Holmes:structural05, Prompter:MSR14}. But none of  these tools presents code example in a structured manner and identify  common features for each code example cluster, which helps developers to reuse the code examples. 

%only recommend code examples in the method level which make them insufficient for  reuse tasks across multiple classes.   We make it one step further to  identify  both common features and alternative features.

\paragraph{Code Reuse}  Although some approaches advocate refactoring code rather than reuse code~\cite{fowler:refactoring}, recent researches have found that these kind of `clone' cannot be easily refactored~\cite{Kim:cloneGenealogy05} and have to be modified to meeting requirements in new context~\cite{Selby:largeReuse05}.  Jigsaw~\cite{Cottrell:jigsaw08} supports small-scale integration of source code into target system  between the example and target context. Based on its ancestor~\cite{Cottrell:generalize07} that identifies structural correspondence based on AST similarity, it greedily matches each element between two contexts, transforms correspondent elements to the target context, and simply copies the source element to the target if it does not correspond with any element in the target. Unfortunately, developer has to provide source and target to enable a one-to-one transformation and resolve all dependencies when pasting code to the target.  Our approach overcomes these two limitations: we extract common functionalities from multiple examples and  identifies how related elements interact with main features.  Other works on code reuse include
Gilligan~\cite{Holmes:reuse07} and Procrustes~\cite{Holmes:ASE09} which try to address the problem of source code integration in the context of medium or large-scale reuse tasks. They automatically suggest program elements that are easy to reuse based on structural relevance and cost of reuse in the source context, and guide users to investigate and plan a non-trivial reuse task. They assume that developers have a perfect example at hand, and they can finish the reuse task by resolving all dependency conflicts and integrating the example to the desired context.  We don't have this assumption and our tool helps users identify the best-fit example. Our idea of leveraging multiple examples to discover commonality is similar to LASE~\cite{LASE:ICSE13}, which applies similar but not identical changes to multiple code locations based on context similarity.   Our approach works in a similar manner of Programing-by-Example in the context of code reuse task. LASE requires users to specify multiple input examples and search for the third one, while our approach uses free-form queries to obtain hundreds of examples and cluster them based on their common features.

%focuses on a task-based code reuse across different methods or even different classes, while LASE is confined to the systematic edit within a single method and requires users to specify all input examples. 


%However, we note that it is not easy to identify a good example as an example is always interleaving with other auxiliary features that should not be integrated. We observe that it is equally difficult, if not more so, to distinguish the major functionality and auxiliary ones from multiple examples than to identify related elements in a pragmatic reuse plan. We target the problem to identify the major features across different reusable examples and leverage Procrustes to evaluate the cost of reuse when recommending the best-fit reusable plan.

\section{Conclusion}\label{sec:conclude}


We present \tool, a novel test-execution-driven approach to synthesize API sequences. Our approach dynamically generates candidates for API sequences during the test execution and backtracks whenever the candidate encounters any test failures or runtime exceptions. Our key novelty is to introduce effective prioritization strategies to divide the search space of APIs to multiple sub-places with prioritized constraints. %\tool explores the actual program behavior with an execution-driven synthesis engine. 
We evaluate \tool on a collection of straight-line programming tasks and a benchmark of non-straight-line subjects derived from open source projects. Our evaluation results indicates that \tool compares generally well with a state-of-the-art API synthesizer and \tool can synthesize the desired program in a practical manner using a few test cases in a reasonable amount of time. We also illustrate that \tool can synthesize APIs in if-conditions and while-loops, which can hardly be handled by existing API synthesis tools.  
%\IEEEpeerreviewmaketitle


% use section* for acknowledgment
%\section{Acknowledgment}

\bibliographystyle{IEEEtran}
\bibliography{synthesis,other}  


% that's all folks
\end{document}


